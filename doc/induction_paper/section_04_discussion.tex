%%%%%%%%%%%%%%%%%%%%%%%%%%%%%%%%%%%%%%%%%%%%%%%%%%%%%%%%%%%%%%%%%%%%%%%%%%%
\section*{Discussion \manuelComment{Manuel's Section!}}

\stephanieComment{General comments: Great attention to the literature. When I read papers I often skip to the discussion before I read the results, so what I like to see in discussions is a fairly detailed summary of the results along with an interpretation of how the results lend insight to the larger context of the field. In addition I think the discussion should address oddities like the apparent change in inflection sharpness in the O3 data. So I think we could use a paragraph or two where we analyze the data a little more thoroughly.}

\griffinComment{I really enjoyed this section but I think it is actually light on discussion of how our specific results impact the field.}

ideas:
\begin{enumerate}
  \item advancements in gene regulation understanding:
    \begin{enumerate}
      \item Davidson's work on the sea urchin gene regulatory network
      \item Classics such as Ackers, Johnson and Shea
      \item Obviously Rob's work
    \end{enumerate}
  \item Building up the complexity of the model, owning different knobs each
  iteration.
  \item MWC as ``one equation to rule them all'' and the pervasive idea of
  allostery in biology.
  \item Significance of the parameters given that we have a zero-parameter fit
  prediction
\end{enumerate}

In the last decade there has been remarkable advancements in our understanding
of how genes are regulated \griffinComment{I would say something sexier than just 'how genes are regulated'. Maybe something like '...advancements in our understanding of how organisms perform the astoundingly complex task of regulating expression' or something}. From the exquisite dissection of the gene regulatory
network of the sea-urchin developmental program [Peters and Davidson] to the
\textit{in vivo} imaging of the central dogma [Ido's MS2, Hernan's MS2, Suntag
paper, Xiaowei recent paper], new genetic reporters and  advancements in
single-cell measurements have taken the field to a new level of precision
measurements traditionally only achievable in more quantitative fields such as
physics and chemistry \talComment{I think this is a great paragraph. I also think it belongs in the intro (or something like it does)}.

This level of precision and the rate at which data is generated calls out for
unifying theoretical frameworks which conceptualize the richness of modern
biological data sets. In this work, we build upon a decade of quantitative
dissection of the input-output function of one of the most abundant genetic
architectures in all domains of life, i.e. simple repression [Mattias paper on
regulonDB]. The \textit{lac} operon in combination with modern synthetic biology
tools have served as an experimental playground on which to test predictive
models of gene regulation \talComment{cite Bintu?}. Statistical mechanical models of gene regulation have
proved to be very effective not only in predicting the mean expression level as a
function of natural parameters such as the transcription factor copy number and
its binding energy to the DNA \cite{Garcia2011} but also in quantifying the effects of DNA looping
\cite{Boedicker2013a} and competing binding sites \cite{Brewster2014} \talComment{Still sounds to me like great intro material. I could be wrong, but the Discussion should focus on the impacts and implications of this paper}.

Interestingly the \textit{lac} repressor was also the first allosteric gene
regulator ever characterized [Monod original paper?]. The story goes that Monod
claimed \talComment{awk} that allostery embodied the ``second secret of life'' \cite{Marzen2013}.
And, as it is so familiar in the physics literature \talComment{I get the idea, but I think the reference to physics will be confusing. I think this sentence could be combined in with the next one}, a single theoretical
framework can explain incredibly diverse phenomena in biology with allostery as
common ground. Ion channels, enzymatic activity \cite{Einav2016}, chromatin
remodeling, bacterial chemotaxis \cite{Endres2006}, and what concerns us in
this paper, gene regulation all can be explained by the Monod-Wyman-Changeux
(MWC) model \cite{Marzen2013}. \stephanieComment{This paragraph adds some nice perspective but I'm not sure it belongs in this section. Maybe we could move it to the introduction. \talComment{+1 Vote from me}}

In this paper we have shown that \talComment{Aha! Sounds like the Discussion section has begun :} the MWC model combined with a statistical
mechanical model of gene expression can provide zero-parameter fit predictions
of the fold-change in gene expression as a function of inducer concentration for
a wide range of transcription factor copy numbers and their binding affinity to
the promoter (see \fref[fig_result2]). We believe that this excellent agreement between
prediction and data is not the result of fitting sigmoidal-like data to a
sigmoidal-like curve \talComment{this sentence is confusing} \nathanComment{related to Stephanie's point on discussing data
a little more thoroughly - especially to address aspects that are not
excellent.}. By taking the equations seriously we were able to rewrite
the fold-change as a function of a highly non-linear parameter we called the
Bohr parameter. \fref[fig_result3] shows that when the data is plotted as a function of this
``natural variable of the system'' \talComment{why quotes?} all the curves \sout{families} collapse into a
single master curve. Our claim is that without the underlying theoretical
framework developed for this problem there is no way one could guess the
functional form of this Bohr parameter in order to show that all the curves
belong to the same family \talComment{Good, but can be tightened}.

\talComment{How about stating explicitly that having such a concrete framework for induction really takes our understanding of transcriptional regulation to the next level.} As the insights on how genes are regulated increase, the theory modularly
increases the physical knobs that can be tuned \talComment{confusing}. These theories must be
contrasted with experiments as we work our way through generating a predictive
theory of gene regulation. In this paper we add yet another piece to this puzzle
by showing how one can own the physical knob represented by the external inducer
concentration. \talComment{This would be a good place to allude to some natural next steps. For example, you could mention looking at deviations from the mean or alluding to the information processing ability of cellular sensing.}
