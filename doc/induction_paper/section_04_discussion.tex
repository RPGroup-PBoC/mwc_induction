\pagebreak
%%%%%%%%%%%%%%%%%%%%%%%%%%%%%%%%%%%%%%%%%%%%%%%%%%%%%%%%%%%%%%%%%%%%%%%%%%%
\section*{Discussion \manuelComment{Manuel's Section!}}

ideas:
\begin{enumerate}
  \item advancements in gene regulation understanding:
    \begin{enumerate}
      \item Davidson's work on the sea urchin gene regulatory network
      \item Classics such as Ackers, Johnson and Shea
      \item Obviously Rob's work
    \end{enumerate}
  \item Building up the complexity of the model, owning different knobs each
  iteration.
  \item MWC as ``one equation to rule them all'' and the pervasive idea of
  allostery in biology.
  \item Significance of the parameters given that we have a zero-parameter fit
  prediction
\end{enumerate}

In the last decade there has been remarkable advancements in our  understanding of how genes are regulated. From the exquisite dissection of the gene regulatory network of the sea-urchin developmental program [Peters & Davidson] to the \textit{in-vivo}  imaging of the central dogma [Ido's MS2, Hernan's MS2, Suntag paper, Xiaowei recent paper], new genetic reporters and  advancements in single-cell measurements has taken the field to a new level of precision measurements traditionally only achievable in more quantitative fields such as physics and chemistry.

This level of precision and the rate at which data is generated calls out for unifying theoretical frameworks under which conceptualize the richness of modern biology data sets. In this work we build upon a decade of quantitative dissection of the input-output function of one of the most abundant genetic architectures in all domains of life, i.e. simple repression [Mattias paper on regulonDB]. The \textit{lac} operon in combination with modern synthetic biology tools have served as an experimental playground on which to test predictive models of gene regulation. Statistical mechanical models of gene regulation have proved to be very effective not only predicting the mean expression level as a function of natural parameters such as the transcription factor copy number and its binding energy to the DNA [Garcia PNAS] but the effect of DNA looping [Boedicker PRL] and competing binding sites [Brewster Cell].

Interestingly the \textit{lac} repressor was also the first allosteric gene regulator ever characterized [Monod original paper?]. The story goes that Monod claimed that allostery embodied the ``second secret of life'' [Marzen 2013]. And, as it is so familiar in the physics literature, a single theoretical framework can explain incredibly diverse phenomena in biology. Ion channels, enzymatic activity, chromatin remodeling, bacterial chemotaxis, and what concerns us in this paper, gene regulation all can be explained by the Monod-Wyman-Changeux (MWC) model [Marzen 2013, Einav 2015, what else?].

In this paper we have shown that the MWC model along a statistical mechanics model of gene expression can provide zero-parameter fit predictions for the fold-change in gene expression as a function of inducer concentration for a wide range of transcription factors copy number and binding affinity to the promoter.
