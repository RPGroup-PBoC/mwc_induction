\pagebreak
%%%%%%%%%%%%%%%%%%%%%%%%%%%%%%%%%%%%%%%%%%%%%%%%%%%%%%%%%%%%%%%%%%%%%%%%%%%
\section*{Results}

\subsection*{The Monod-Wyman-Changeux (MWC) Model \talComment{Tal's Section!}} 

This paper builds upon an extensive dialogue between theory and experiment in
transcriptional regulation. A first round of experiments measured the dependence
of gene expression on repressor copy number and binding strength
\cite{Garcia2011}. A second round of experiments pushed beyond the ``independent
promoter approximation'' to acknowledge the fact that different genes compete
for the same regulatory apparatus. Here too we were able to show that
theoretical predictions of this subtle effect were consistent with their
measured counterparts, even permitting the collapse of data from multiple
experiments onto master curves \cite{Brewster2014, Weinert2014}. In the current
paper, we consider yet another layer of complexity by analyzing signaling in the
context of transcription.

The ability of the the Lac repressor
to regulate transcription has been previously characterized by an equilibrium
model where the probability of each state of repressor and RNA polymerase
occupancy is proportional to its Boltzmann weight \cite{Daber2011a,
	Phillips2015a}. We begin with a summary of this model. Suppose there are \(P\)
RNA polymerase (RNAP) and \(R\) repressor molecules in a cell. \(R_A\)
repressors will be in the active state (the favored state when repressor is not
bound to inducer; in this state the repressor binds tightly to DNA) and the
remaining \(R_I\) repressors will be in the inactive state (the predominant
state when repressor is bound to inducer; in this state the repressor binds
weakly to DNA) so that \(R_A+R_I=R\).

We first model the interaction between the Lac repressor and DNA by enumerating
all possible states and their corresponding weights. As shown in
\fref[figpolymeraseRepressorStates], the Lac promoter can either be empty,
occupied by RNAP, or occupied by either an active or inactive repressor
molecule. Assume that there are $N_{\text{NS}}$ non-specific sites on the DNA outside
the Lac operator where RNAP or the Lac repressor can bind. \(\Delta\varepsilon_{PD}\)
represents the energy difference between RNAP bound to the Lac operator or bound
elsewhere on the DNA; \(\Delta\varepsilon_{RD,A}\) and \(\Delta\varepsilon_{RD,I}\) equal the
difference in energy when the Lac repressor is bound to the Lac operator
compared to when it is bound non-specifically elsewhere on the DNA in the active
and inactive state, respectively. $\beta = \frac{1}{k_BT}$ where $k_B$ is
Boltzmann's constant and $T$ is the temperature of the system.

\begin{figure}[h]
	\centering \includegraphics[scale=\globalScalePlots]{figure1v3.pdf}
	\caption{{\bf States and weights for simple repression.} Both RNAP (light blue)
		and repressor compete for DNA binding. There are $R_A$ repressors in the active
		state (green) and $R_I$ repressors in the inactive state (red), with the latter
		type typically bound to inducer (gold). The difference in energy between a
		repressor bound to the Lac operator and to another non-specific site on the DNA
		equals $\Delta\epsilon_{RD,A}$ in the active state and $\Delta\epsilon_{RD,I}$
		in the inactive state; the $P$ RNAP have a corresponding energy difference
		$\Delta\epsilon_{PD}$. The number of active repressors $R_A$ includes
		repressors that are unbound, singly bound, or doubly bound to inducer, although
		the majority of active state repressors will not be bound to inducer (which
		pushes them into the inactive state). Similarly, the $R_I$ term includes all
		inactive state repressors bound to any number of inducer molecules, with the
		most prevalent state shown in the figure. The factor of 2 in the repressor
		weights comes from the ability of either repressor dimer to bind to DNA. }
	\label{figpolymeraseRepressorStates}
\end{figure}

In thermodynamic models of transcription, gene expression is proportional to the
probability $p_{\text{bound}}$ that RNAP is bound to the Lac operator which is
given by
\begin{equation}\label{eq2}
p_{\text{bound}}(R)=\frac{p}{1+2r_A+2r_I+p},
\end{equation}
where
\begin{align}
p &= \frac{P}{N_{\text{NS}}}e^{-\beta  \Delta\varepsilon_{PD}} \\
r_A &= \frac{R_A}{N_{\text{NS}}}e^{-\beta \Delta\varepsilon_{RD,A}} \\
r_I &= \frac{R_I}{N_{\text{NS}}}e^{-\beta  \Delta\varepsilon_{RD,I}}.
\end{align}
Note the factor of 2 in the repressor states, which arises from the possibility
that either repressor dimer may bind to DNA. Gene expression can be readily
measured experimentally by exploiting the fold-change,
\begin{equation}\label{eq3}
\foldchange\equiv \frac{p_{\text{bound}}(R)}{p_{\text{bound}}(0)}=\frac{1+p}{1+2r_A+2r_I+p}.
\end{equation}
We can simplify this expression using two well-justified approximations: \(p\ll
1\) and \(r_I\ll r_A\). The first approximation is called the weak promoter
approximation and is valid for the wild type Lac promoter \cite{Brewster2012}.
The second approximation follows because $e^{-\beta  \Delta\varepsilon_{RD,I}}$ is
approximately 1000 times smaller than $e^{-\beta \Delta\varepsilon_{RD,A}}$ for
the Lac repressor \cite{Daber2011a}. Using these approximations, the fold-change
reduces to the form
\begin{equation}\label{eq4}
\foldchange\approx \frac{1}{1+2r_A}=\left(1+\frac{2R_A}{N_{\text{NS}}}e^{-\beta  \Delta\varepsilon_{RD,A}}\right)^{-1}.
\end{equation}

We now introduce the role of inducer binding, which changes the number of
repressors in the active and inactive allosteric states. We begin by defining
\(p_A(c) \equiv \frac{R_A(c)}{R}\) to be the fraction of repressors in the
active state given a concentration \(c\) of the inducer IPTG, so that %We define
% \(V\) as the volume
%of an \textit{E. coli} cell, \([R]=\frac{R}{V}\) as the concentration of
%repressors, and \(\K=\frac{N_{\text{NS}}}{V}e^{\beta
% \Delta\varepsilon_{RD,A}}\)
%as the dissociation constant of the active repressor binding to the Lac
% operator
%DNA. This last expression, which links the physical energies of the system with
%the language of dissociation constants and chemical rates, is discussed in
%detail in the Supplementary Information.
\eref[eq4] becomes
\begin{equation}
\foldchange = \left( 1+p_A(c) \frac{2 R}{N_{\text{NS}}}e^{-\beta 
	\Delta\varepsilon_{RD,A}} \right)^{-1}. \label{eq5}
\end{equation}
%\begin{align}
%\foldchange &= \left( 1+\frac{2p_A(c) [R] V}{N_{\text{NS}}}e^{-\beta  \Delta\varepsilon
%	_{RD,A}} \right)^{-1} \nonumber \\
%&= \left( 1+\frac{2p_A(c) [R]}{\K} \right)^{-1}. \label{eq5}
%\end{align}

The Lac repressor is a tetrameric protein, a dimer of dimers, with four
identical binding sites for an inducer. It is unknown whether the two dimers are
allosterically independent (i.e. whether one dimer can be active while the other
is inactive) or allosterically linked (with both dimers being either
simultaneously active or simultaneously inactive). While no direct measurements
have as of yet been carried out to definitively distinguish between these two
models, in Appendix \ref{AppendixAllostery} we describe a simple experiment
which can do precisely this. In this paper, we proceed with the assumption that
the two repressor dimers are allosterically independent.

\begin{figure}[h]
	\centering \includegraphics[scale=\globalScalePlots]{figure2v2.pdf}
	\caption{{\bf The eight states of a Lac repressor dimer.} The Lac repressor has
		an active conformation (green, left column) and inactive conformation (red,
		right column), with the energy difference between these two states given by
		$\varepsilon$. In each conformation, the repressor can bind an inducer (gold) at
		two sites. Each state is shown with its corresponding Boltzmann weight. The top
		dimer can be in the active or inactive state independent of the bottom dimer.
		The top dimer is drawn partly opaque because its different states will not
		effect the probability that the bottom dimer is active. %		If the sum of the
		% active state weights shown (bottom left) is greater than the sum of
		%		the inactive state weights (bottom right), the repressor is more likely to be
		%		in the active state.
	} \label{figrepressorInducerStates}
\end{figure}

As shown in \fref[figrepressorInducerStates], we can enumerate the relative
likelihood of the eight possible conformations of a repressor dimer (the dimer
can be in an active or inactive state, and each of its two inducer binding sites
can be empty or occupied), using the difference in energy $\varepsilon$ between a
Lac repressor dimer in the active and inactive state. From these eight states, we can
compute the probability \(p_A(c)\) that a dimer will be in the active state
as the sum of the weights of the active states divided by the sum of the weights
of every possible state, namely,
\begin{equation}\label{eq6}
p_A(c)=\frac{\left(1+\frac{c}{K_A}\right)^2}{\left(1+\frac{c}{K_A}\right)^2+e^{-\beta  \varepsilon }\left(1+\frac{c}{K_I}\right)^2}.
\end{equation}

%Note that in the \fref[figrepressorInducerStates] states and weights, we assumed
%that within the Lac tetramer, each Lac dimer could be active or inactive
%independently of the other dimer. An alternative model presumes that . While no
%direct experimental measurements have been yet been carried out to definitively
%distinguish between these two models, in Appendix \ref{AppendixAllostery} we
%describe a simple experiment where upon removing the tetramerization region of
%the Lac repressor we can distinguish between these two models. \talComment{Rob,
%	this is super cool!!! If you agree with the theory in this Appendix, let us know
%	if you approve of us carrying out this experiment!}

It is hypothesized that in the absence of inducer ($c=0$), all of the repressors
are present in the active state, which implies $\beta \varepsilon \gg 1$. As
discussed in Appendix \ref{AppendixModel}, this assertion seems supported by the
available data, but must ultimately be validated by direct measurement, as is
possible by NMR \cite{Gardino2003, Boulton2016}. Given $\beta \varepsilon \gg
1$, we can approximate the term $e^{-\beta \varepsilon}
\left(1+\frac{c}{K_I}\right)^2 \approx e^{-\beta  \varepsilon
}\left(\frac{c}{K_I}\right)^2$ in the denominator of \eref[eq6]. To see this,
note that both expressions are negligible compared to
$\left(1+\frac{c}{K_A}\right)^2$ for small $c$; on the other hand, the term
$e^{-\beta \varepsilon} \left(1+\frac{c}{K_I}\right)^2$ becomes non-negligible
once $e^{-\beta \varepsilon }\left(\frac{c}{K_I}\right)^2 \gtrsim 1$, in which
case $\frac{c}{K_I} \gg 1$ so that our approximation is again valid. Therefore,
we can approximate the probability that a repressor dimer is active as
\begin{equation}\label{eq6v2}
p_A(c) \approx \frac{\left(1+\frac{c}{K_A}\right)^2}{\left(1+\frac{c}{K_A}\right)^2+e^{-\beta  \varepsilon }\left(\frac{c}{K_I}\right)^2} \equiv \frac{\left(1+\frac{c}{K_A}\right)^2}{\left(1+\frac{c}{K_A}\right)^2+\left(\frac{c}{\KIeff}\right)^2},
\end{equation}
where we have introduced the effective dissociation constant $\KIeff = K_I
e^{\beta  \varepsilon/2}$. Substituting this result into \eref[eq5] yields the
complete formula
\begin{equation}\label{eq7}
\foldchange= \left(
1+\frac{\left(1+\frac{c}{K_A}\right)^2}{\left(1+\frac{c}{K_A}\right)^2+\left(\frac{c}{\KIeff}\right)^2}\frac{2 R}{N_{\text{NS}}}e^{-\beta \Delta\varepsilon_{RD,A}} \right)^{-1},
\end{equation}
%\begin{equation}\label{eq7}
%\foldchange= \left(
%1+\frac{\left(1+\frac{c}{K_A}\right)^2}{\left(1+\frac{c}{K_A}\right)^2+\left(\frac{c}{\KIeff}\right)^2}\frac{2[R]}{\K} \right)^{-1},
%\end{equation}
which predicts that given a concentration \(c\) of the inducer IPTG, \(R\)
copies of the Lac repressor, and the $N_{\text{NS}} = 4.6 \times 10^6$
non-specific binding sites on the \textit{E. coli} genome, the fold-change in
gene expression will depend solely on 3 parameters: the DNA binding affinity of
the repressor ($\Delta\varepsilon_{RD,A}$) and the inducer binding affinities
for the repressor in the active state (\(K_A\)) and inactive state (\(\KIeff\)),
with this latter quantity also incorporating the difference in free energy
between the active and inactive states of the repressor.