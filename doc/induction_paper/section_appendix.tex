%%%%%%%%%%%%%%%%%%%%%%%%%%%%%%%%%%%%%%%%%%%%%%%%%%%%%%%%%%%%%%%%%%%%%%%%%%%
%%%%%%%%%%%%%%%%%%%%%%%%%%%%%   APPENDIX   %%%%%%%%%%%%%%%%%%%%%%%%%%%%%%%%
%%%%%%%%%%%%%%%%%%%%%%%%%%%%%%%%%%%%%%%%%%%%%%%%%%%%%%%%%%%%%%%%%%%%%%%%%%%
\appendix

%%%%%%%%%%%%%%%%%%%%%%%%%%%%%%%%%%%%%%%%%%%%%%%%%%%%%%%%%%%%%%%%%%%%%%%%%%%
\section{Fold-Change Model} \label{AppendixModel}

\talComment{Discussion of $p_{active} = \frac{1+\frac{O}{N_{NS}}e^{-\beta
			\varepsilon_{\text{DNA}}}}{1+\frac{O}{N_{NS}}e^{-\beta \varepsilon_{\text{DNA}}} +
		e^{-\beta \varepsilon}}$ (although possibly in $K_D$ language to be consistent?) as
	well as Hernan's actual DNA binding measurements $e^{-\beta
		\varepsilon_{\text{DNA}^{\text{measured}}}} = \frac{1}{1+e^{-\beta\varepsilon}}
	e^{-\beta \varepsilon_{\text{DNA}}}$}

Make some graphs showing how many $\varepsilon$ values (given Brewster/Franz data)
all have the same fit residuals and hence no value of $\varepsilon$ can be
distinguished. In addition, need to mention the sloppiness issue in Appendix \ref{AppendixSloppiness}, so that a value of $\varepsilon$ has to be decided outside of the fitting mechanism. NMR measurements will settle the question.

\pagebreak
%%%%%%%%%%%%%%%%%%%%%%%%%%%%%%%%%%%%%%%%%%%%%%%%%%%%%%%%%%%%%%%%%%%%%%%%%%%
\section{Allostery within the Lac Repressor} \label{AppendixAllostery}

\fref[figrepressorInducerStates] in the main text shows the possible states and
weights for a Lac repressor tetramer assuming that both dimers within the Lac
repressor can independently be active or inactive. Using this model, the
probability that one of the dimers is in the active state is given by
\begin{equation}\label{AppendixEq1}
p_A^{\text{dimer}}(c)=\frac{\left(1+\frac{c}{K_A}\right)^2}{\left(1+\frac{c}{K_A}\right)^2+e^{-\beta  \varepsilon }\left(1+\frac{c}{K_I}\right)^2},
\end{equation}
which yields the formula for fold-change,
\begin{equation}\label{AppendixEq2}
\foldchange= \left(
1+\frac{\left(1+\frac{c}{K_A}\right)^2}{\left(1+\frac{c}{K_A}\right)^2+e^{-\beta  \varepsilon }\left(1+\frac{c}{K_I}\right)^2}\frac{2 R}{N_{\text{NS}}}e^{-\beta \Delta\varepsilon_{RD,A}} \right)^{-1}.
\end{equation}
%\begin{equation}\label{AppendixEq2}
%\foldchange= \left(
%1+\frac{\left(1+\frac{c}{K_A}\right)^2}{\left(1+\frac{c}{K_A}\right)^2+e^{-\beta  \varepsilon }\left(1+\frac{c}{K_I}\right)^2}\frac{2[R]}{\K} \right)^{-1}.
%\end{equation}
Now suppose that the tetramerization region within the Lac repressor gene is
removed, which creates a functional dimeric repressor that: (1) can bind to DNA;
(2) exists in both an active and inactive allosteric conformation; and (3) has
two binding sites for the inducer IPTG \cite{Daber2011a, Daber2009}. This
construct would have the same states and weights shown in
\fref[figrepressorInducerStates], so that its probability of being active is
still given by \eref[AppendixEq1]. Furthermore, since the Lac repressor
ribosomal binding site has not been modified, there would now be twice as many
Lac repressor dimers as there were Lac repressor tetramers; however, whereas
each tetramer could originally bind to DNA in two configurations (with each of
its dimers), the now-disjoint Lac repressor dimers can only bind to the DNA in
one configuration. These two factors cancel each other, so that the states and
weights for the dimeric repressor binding to DNA would be identical to
\fref[figpolymeraseRepressorStates] and the fold-change equation would still be
given by \eref[AppendixEq2]. In other words, within this model where the Lac
repressor tetramer consists of two dimers which can be independently active or
inactive, fold-change measurements will not be affected at all by removing the
tetramerization region. Note that throughout this analysis, we have assumed that
removing the tetramerization region does not alter the thermodynamic parameter
$K_A$, $K_I$, and $\varepsilon$.

We now turn to a second model of the Lac tetramer, where the two dimers must
either simultaneously active or simultaneously inactive. In other words, the
repressor as a whole is either active or inactive. In such a case, the Lac
repressor can be viewed as an allosteric receptor with four identical inducer
binding sites, which implies that the probability that the Lac repressor is
active is given by
\begin{equation}\label{AppendixEq3}
p_A^{\text{tetramer}}(c)=\frac{\left(1+\frac{c}{K_A}\right)^4}{\left(1+\frac{c}{K_A}\right)^4+e^{-\beta  \varepsilon }\left(1+\frac{c}{K_I}\right)^4}.
\end{equation}
The corresponding formula for fold-change will now have these fourth powers,
\begin{equation}\label{AppendixEq4}
\foldchange= \left(
1+\frac{\left(1+\frac{c}{K_A}\right)^4}{\left(1+\frac{c}{K_A}\right)^4+e^{-\beta  \varepsilon }\left(1+\frac{c}{K_I}\right)^4}\frac{2 R}{N_{\text{NS}}}e^{-\beta \Delta\varepsilon_{RD,A}} \right)^{-1}.
\end{equation}
%\begin{equation}\label{AppendixEq4}
%\foldchange= \left(
%1+\frac{\left(1+\frac{c}{K_A}\right)^4}{\left(1+\frac{c}{K_A}\right)^4+e^{-\beta  \varepsilon }\left(1+\frac{c}{K_I}\right)^4}\frac{2[R]}{\K} \right)^{-1}.
%\end{equation}
Now suppose that the tetramerization region was removed within this model. Once
again, there would be twice as many dimers, each able to bind to DNA in only one
configuration, so that these two factors cancel each other. But now each dimers
must necessarily be active or inactive independently of all other dimers, and
therefore the probability of a repressor being active and the corresponding
equation fold-change would be given by \eref[AppendixEq1][AppendixEq2],
respectively. Upon changing the fourth powers to second powers, we expect that
the fold-change curves will dramatically shift.

Thus, by comparing fold-change measurements of a Lac repressor with and without
the tetramerization region, we can immediately distinguish between these two
models. \fref[figAllosteryModels1]\letter{A} shows fold-change data for the O2
operator. As in the main text, we use the RBS 1027 data set to determine the
thermodynamic parameters and then predict the remaining data sets with no
further fitting. After removing the tetramerization region, we would predict
that the fold-change curves shown in \fref[figAllosteryModels1]\letter{B} would
result. In contrast, if the independent dimer model is correct, we would expect
the fold-change curves to remain stationary when the tetramerization region is
removed. These plots demonstrate the need to use theoretical models not merely
to characterize known data, but to \textit{predict} the results of novel
experiments.

\begin{figure}[h]
	\centering \includegraphics[scale=\globalScalePlots]{SIfigure4.pdf}
	\caption{{\bf Effects of removing tetramerization region of Lac repressor.}
		\letterParen{A} Fold-change of tetrameric Lac repressor binding to the O2
		operator. Assuming that both repressor dimers are simultaneously active or
		inactive, \eref[AppendixEq4], the best fit parameters are $K_A = 45 \times
		10^{-6}\,\text{M}$, $K_I = 3 \times 10^{-6}\,\text{M}$, and $\beta \varepsilon
		= 4.5$. While these thermodynamic parameters are different from the
		corresponding values found from the independent dimer model,
		\eref[AppendixEq2], the resulting curves fit the data well. \letterParen{B}
		Once the tetramerization region is removed, the predicted fold-change is given
		by \eref[AppendixEq2] using these same thermodynamic parameters. The new
		fold-change curves are markedly different from the original curves. The opaque
		dots \textit{do not} represent data \talComment{not yet!}, but only server as a
		reference point for the theoretical curves. } \label{figAllosteryModels1}
\end{figure}

Another interesting aspect of removing the tetramerization region is that the
shift in the fold-change curves depends upon the $\varepsilon$ parameter.
\fref[figAllosteryModels2] shows the corresponding predictions assuming that
the repressor dimers are not independent and using the larger value $\beta
\varepsilon = 0$. In this case, the corresponding shift upon removing the
tetramerization is noticeably smaller. Hence, if a shift is observed in the
fold-change curves, the size of the shift would be an indirect measurement of
$\varepsilon$.

\begin{figure}[h]
	\centering \includegraphics[scale=\globalScalePlots]{SIfigure3.pdf}
	\caption{{\bf Dependence of tetramerization region removal on $\boldsymbol{\varepsilon}$.}
		\letterParen{A} As in \fref[figAllosteryModels1], we fit fold-change of Lac repressor with a tetramerization region binding to the O2 operator, but this time forcing $\beta \varepsilon = 0$. Using \eref[AppendixEq4], the best fit parameters are $K_A = 60 \times 10^{-6}\,\text{M}$, $K_I = 10 \times 10^{-6}\,\text{M}$, and $\beta \varepsilon = 0$. \letterParen{B} With this smaller $\beta \varepsilon$ value, the corresponding theoretical predictions exhibit a much larger fold-change than those in \fref[figAllosteryModels1]\letter{B}.
	}
	\label{figAllosteryModels2}
\end{figure}

%%%%%%%%%%%%%%%%%%%%%%%%%%%%%%%%%%%%%%%%%%%%%%%%%%%%%%%%%%%%%%%%%%%%%%%%%%%
\section{Sloppiness} \label{AppendixSloppiness}

\talComment{An initial discussion of sloppiness and the $\varepsilon$ parameter. Needs to be updated with one of our actual data sets. Perhaps if we have enough data sets this sloppiness will go away?}

We have all seen by now that we can wiggle the best fit parameters around and still get good fitting. To better visualize the problem, I took Stephanie's 2016-06-18 data for HG 104 and RBS 1027 and fit it to the form \eref[AppendixSloppinessEq1], forcing the $\beta \varepsilon_R$ to vary between -10 and 10. \fref[SIfig1] shows the results.
\begin{figure}[h]
	\centering \includegraphics{SIfigure1.pdf}
	\caption{{\bf Many best fit parameters generate good fits.} \letterParen{A} I forced $\beta \varepsilon_R$ to take on a particular value and fit $K_A$ and $K_I$ to the data using \eref[AppendixSloppinessEq1]. All of the resulting fit curves look good. \letterParen{B} The best fit $K_A$ and $K_I$ values. This graph seemed quite provacative to me, since the values of $K_A$ and $K_I$ seem to be linear curves on this logarithmic plot.}
	\label{SIfig1}
\end{figure}

Rather than continuing to beat around the bush, I thought it was time to tackle this issue head on. The following analysis is for the following $n=2$
exponent form,
\begin{align} \label{AppendixSloppinessEq1}
\foldchange &= \frac{1}{1+\frac{\left( 1 + \frac{c}{K_A} \right)^2 \left(1 + e^{-\beta \varepsilon_R} \right)}{\left( 1 + \frac{c}{K_A} \right)^2 + e^{-\beta \varepsilon_R} \left( 1 + \frac{c}{K_I} \right)^2} \frac{2 \Rtot}{N_{NS}} e^{-\beta \Delta\varepsilon_{\text{O2}}^{(\text{Hernan})}}}\nonumber\\
&\equiv \frac{1}{1+\frac{\left( 1 + \frac{c}{K_A} \right)^2 \left(1 + e^{-\beta \varepsilon_R} \right)}{\left( 1 + \frac{c}{K_A} \right)^2 + e^{-\beta \varepsilon_R} \left( 1 + \frac{c}{K_I} \right)^2} r},
\end{align}
where for notational convenience I have defined $r = \frac{2 \Rtot}{N_{NS}}
e^{-\beta \Delta\varepsilon_{\text{O2}}^{(\text{Hernan})}}$ (with $r \approx 1$ for
HG 104, using Hernan's values). Recall that $K_I < K_A$, since the inducer makes
the repressor more likely to assume the inactive state. I will analyze this
fold-change equation in the two limits: $e^{-\beta \varepsilon_R} \ll 1$ and
$e^{-\beta \varepsilon_R} \gg 1$.

\subsection{$\boldsymbol{e^{-\beta \varepsilon_R} \ll 1}$}

In the limit $e^{-\beta \varepsilon_R} \ll 1$, the fold-change equation becomes
\begin{align} \label{AppendixSloppinessEq2}
\foldchange &= \frac{1}{1+\frac{\left( 1 + \frac{c}{K_A} \right)^2}{\left( 1 + \frac{c}{K_A} \right)^2 + e^{-\beta \varepsilon_R} \left(\frac{c}{K_I} \right)^2} r} \nonumber\\
&\equiv \frac{1}{1+\frac{\left( 1 + \frac{c}{K_A} \right)^2}{\left( 1 + \frac{c}{K_A} \right)^2 + \left(\frac{c}{\widetilde{K}_I} \right)^2} r}.
\end{align}
where we have defined $\widetilde{K}_I = K_I e^{\beta \varepsilon_R/2}$. This
suggests that what we are actually fitting $K_A$ and $\widetilde{K}_I$ in this
regime, as confirmed by \fref[SIfig2] in the region $e^{-\beta \varepsilon_R} \ll 1$ (note that the $K_I e^{\beta \varepsilon_R/2}$ values do not change).
%where we have made two approximations. First, we assumed $\left(1 + e^{-\beta
% \varepsilon_R} \right) \approx 1$ in the numerator. Second, we used $\left( 1 +
% \frac{c}{K_A} \right)^2 + e^{-\beta \varepsilon_R} \left( 1 + \frac{c}{K_I}
% \right)^2 \approx \left( 1 + \frac{c}{K_A} \right)^2 + e^{-\beta \varepsilon_R}
% \left(\frac{c}{K_I} \right)^2$ in the denominator because the $e^{-\beta
% \varepsilon_R} \left(1 + \frac{c}{K_I} \right)^2$ term
\begin{figure}[h]
	\centering \includegraphics{SIfigure2.pdf} \caption{{\bf The altered
			form of $K_I$ we are actually fitting when $\boldsymbol{e^{-\beta \varepsilon_R}
				\ll 1}$.} Redrawing of \fref[SIfig1]\letter{B} showing $K_I$ and $\widetilde{K}_I
		= K_I e^{\beta \varepsilon_R/2}$, the latter forming a straight line when
		$e^{-\beta \varepsilon_R} \ll 1$.} \label{SIfig2}
\end{figure}

With only two effective parameters, you should be able to figure out their
values without doing any fitting. To do this, consider three important
characteristics of a titration curve: (1) the \textit{leakiness} equals the fold-change
when $c=0$; (2) the \textit{dynamic range} equals the difference between the maximal
fold-change at $c\to\infty$ and $c=0$; and $[EC_{50}]$ equals the concentration at
which fold-change reaches halfway between its minimum and maximum values.

Using \eref[AppendixSloppinessEq1], 
\begin{align}
\lim_{c = 0}\foldchange &= \frac{1}{1+r}\\
\lim_{c \to \infty}\foldchange &= \frac{\left(\frac{K_A}{\widetilde{K}_I}\right)^2 + 1}{\left(\frac{K_A}{\widetilde{K}_I}\right)^2 + 1+r}
\end{align}
so that
\begin{align}
\text{leakiness} &= \frac{1}{1+r} \label{AppendixSloppinessEq3} \\
\text{dynamic range} &= \frac{\left(\frac{K_A}{\widetilde{K}_I}\right)^2 + 1}{\left(\frac{K_A}{\widetilde{K}_I}\right)^2 + 1+r} - \frac{1}{1+r}. \label{AppendixSloppinessEq4}
\end{align}
Since you can read the leakiness and dynamic range straight off the data, \eref[AppendixSloppinessEq4] already gives us one relationship for $K_A$ and $\widetilde{K}_I$,
\begin{equation} \label{AppendixSloppinessEq5}
\frac{K_A}{\widetilde{K}_I} = \left( \left(1+r\right) \frac{\text{dynamic range} - \frac{1}{1+r}}{1-\text{dynamic range}} \right)^{1/2}.
\end{equation}

We next solve for $[EC_{50}]$, showing first an exact formula and then an approximation that can be achieved by noting that $\widetilde{K}_I < K_A$ and that $r \approx 1$,
\begin{align}
[EC_{50}] &= \frac{K_A \widetilde{K}_I^2 \left(1+r\right) + \sqrt{K_A^2 \widetilde{K}_I^2 \left(1+r\right) \left(K_A^2 + 2 \widetilde{K}_I^2 \left(1+r\right) \right)}}{K_A^2 + \widetilde{K}_I^2 \left(1+r\right)}\nonumber\\
&\approx \widetilde{K}_I \left(1+r\right)^{1/2}.
\end{align}
Combining this formula with \eref[AppendixSloppinessEq5], we can directly find the value of our two parameters without resorting to fitting,
\begin{align}
\widetilde{K}_I &= \frac{[EC_{50}]}{\left(1+r\right)^{1/2}}\\
K_A &= [EC_{50}] \left(\frac{\text{dynamic range} - \frac{1}{1+r}}{1-\text{dynamic range}} \right)^{1/2}.
\end{align}


\subsection{$\boldsymbol{e^{-\beta \varepsilon_R} \gg 1}$}

In this limit, \eref[AppendixSloppinessEq1] becomes 
\begin{equation}
\foldchange = \frac{1}{1+\left(\frac{1 + \frac{c}{K_A}}{1 + \frac{c}{K_I}}\right)^2 r},
\end{equation}
since in the numerator $\left( 1 + e^{-\beta \varepsilon_R} \right) \approx
e^{-\beta \varepsilon_R}$ while in the denominator the
$\left(1+\frac{c}{K_A}\right)^2$ term is always smaller than the $e^{-\beta
	\varepsilon_R} \left(1+\frac{c}{K_I}\right)^2$ term. In this case, we are also
fitting only two parameters (only $K_A$ and $K_I$), and we see that the $\beta
\varepsilon_R$ value should not influence these parameters, as is confirmed in
\fref[SIfig1]\letter{B}.

You can play the same game as in the previous section to get the values of these two parameters from the dynamic range and $[EC_{50}]$. For now, I skip the details and just write the results,
\begin{align}
K_I &= \frac{[EC_{50}]}{\left(2+r\right)^{1/2}-1}\\
K_A &= [EC_{50}] \frac{r^{1/2}}{\left(2+r\right)^{1/2}-1} \left(\frac{\text{dynamic range}}{1-\text{dynamic range}}\right).
\end{align}

\subsection{Thoughts}

It appears that in either regime, we are really fitting two parameters, and a
third parameter is ``going along for the ride.'' I am not sure how to determine
whether $e^{-\beta \varepsilon_R} \ll 1$ or $e^{-\beta \varepsilon_R} \gg 1$, since
both regimes yield great (and very similar) fits. However, within the two
regimes I now think that I understand exactly what $K_A$ and $K_I$ parameters
will come out of fitting. We need another insight to get us the sign of $\beta
\varepsilon$, and then I think the sloppiness in our best-fit parameters will be
much more under control.