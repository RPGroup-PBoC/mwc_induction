%%%%%%%%%%%%%%%%%%%%%%%%%%%%%%%%%%%%%%%%%%%%%%%%%%%%%%%%%%%%%%%%%%%%%%%%%%%
%%%%%%%%%%%%%%%%%%%%%%%%%%%%%   APPENDIX   %%%%%%%%%%%%%%%%%%%%%%%%%%%%%%%%
%%%%%%%%%%%%%%%%%%%%%%%%%%%%%%%%%%%%%%%%%%%%%%%%%%%%%%%%%%%%%%%%%%%%%%%%%%%
\appendix

%%%%%%%%%%%%%%%%%%%%%%%%%%%%%%%%%%%%%%%%%%%%%%%%%%%%%%%%%%%%%%%%%%%%%%%%%%%
\section{Fold-Change Model} \label{AppendixModel}

\talComment{Discussion of $p_{active} = \frac{1+\frac{O}{N_{NS}}e^{-\beta
			\varepsilon_{\text{DNA}}}}{1+\frac{O}{N_{NS}}e^{-\beta \varepsilon_{\text{DNA}}} +
		e^{-\beta \varepsilon}}$ (although possibly in $K_D$ language to be consistent?) as
	well as Hernan's actual DNA binding measurements $e^{-\beta
		\varepsilon_{\text{DNA}^{\text{measured}}}} = \frac{1}{1+e^{-\beta\varepsilon}}
	e^{-\beta \varepsilon_{\text{DNA}}}$}

Make some graphs showing how many $\varepsilon$ values (given Brewster/Franz data)
all have the same fit residuals and hence no value of $\varepsilon$ can be
distinguished. In addition, need to mention the sloppiness issue in Appendix \ref{AppendixSloppiness}, so that a value of $\varepsilon$ has to be decided outside of the fitting mechanism. NMR measurements will settle the question.

%%%%%%%%%%%%%%%%%%%%%%%%%%%%%%%%%%%%%%%%%%%%%%%%%%%%%%%%%%%%%%%%%%%%%%%%%%%
\section{Allostery within the Lac Repressor} \label{AppendixAllostery}

\fref[figrepressorInducerStates] in the main text shows the possible states and
weights for a Lac repressor tetramer assuming that both dimers within the Lac
repressor can independently be active or inactive. Using this model, the
probability that one of the dimers is in the active state is given by
\begin{equation}\label{AppendixEq1}
p_A^{\text{dimer}}(c)=\frac{\left(1+\frac{c}{K_A}\right)^2}{\left(1+\frac{c}{K_A}\right)^2+e^{-\beta  \varepsilon }\left(1+\frac{c}{K_I}\right)^2},
\end{equation}
which yields the formula for fold-change,
\begin{equation}\label{AppendixEq2}
\foldchange= \left(
1+\frac{\left(1+\frac{c}{K_A}\right)^2}{\left(1+\frac{c}{K_A}\right)^2+e^{-\beta  \varepsilon }\left(1+\frac{c}{K_I}\right)^2}\frac{2 R}{N_{\text{NS}}}e^{-\beta \Delta\varepsilon_{RD,A}} \right)^{-1}.
\end{equation}
%\begin{equation}\label{AppendixEq2}
%\foldchange= \left(
%1+\frac{\left(1+\frac{c}{K_A}\right)^2}{\left(1+\frac{c}{K_A}\right)^2+e^{-\beta  \varepsilon }\left(1+\frac{c}{K_I}\right)^2}\frac{2[R]}{\K} \right)^{-1}.
%\end{equation}
Now suppose that the tetramerization region within the Lac repressor gene is
removed, which creates a functional dimeric repressor that: (1) can bind to DNA;
(2) exists in both an active and inactive allosteric conformation; and (3) has
two binding sites for the inducer IPTG \cite{Daber2011a, Daber2009}. This
construct would have the same states and weights shown in
\fref[figrepressorInducerStates], so that its probability of being active is
still given by \eref[AppendixEq1]. Furthermore, since the Lac repressor
ribosomal binding site has not been modified, there would now be twice as many
Lac repressor dimers as there were Lac repressor tetramers; however, whereas
each tetramer could originally bind to DNA in two configurations (with each of
its dimers), the now-disjoint Lac repressor dimers can only bind to the DNA in
one configuration. These two factors cancel each other, so that the states and
weights for the dimeric repressor binding to DNA would be identical to
\fref[figpolymeraseRepressorStates] and the fold-change equation would still be
given by \eref[AppendixEq2]. In other words, within this model where the Lac
repressor tetramer consists of two dimers which can be independently active or
inactive, fold-change measurements will not be affected at all by removing the
tetramerization region. Note that throughout this analysis, we have assumed that
removing the tetramerization region does not alter the thermodynamic parameter
$K_A$, $K_I$, and $\varepsilon$.

We now turn to a second model of the Lac tetramer, where the two dimers must
either be simultaneously active or simultaneously inactive. In other words, the
repressor as a whole is either active or inactive. In such a case, the Lac
repressor can be viewed as an allosteric receptor with four identical inducer
binding sites, which implies that the probability that the Lac repressor is
active is given by
\begin{equation}\label{AppendixEq3}
p_A^{\text{tetramer}}(c)=\frac{\left(1+\frac{c}{K_A}\right)^4}{\left(1+\frac{c}{K_A}\right)^4+e^{-\beta  \varepsilon }\left(1+\frac{c}{K_I}\right)^4}.
\end{equation}
The corresponding formula for fold-change will now have these fourth powers,
\begin{equation}\label{AppendixEq4}
\foldchange= \left(
1+\frac{\left(1+\frac{c}{K_A}\right)^4}{\left(1+\frac{c}{K_A}\right)^4+e^{-\beta  \varepsilon }\left(1+\frac{c}{K_I}\right)^4}\frac{2 R}{N_{\text{NS}}}e^{-\beta \Delta\varepsilon_{RD,A}} \right)^{-1}.
\end{equation}
%\begin{equation}\label{AppendixEq4}
%\foldchange= \left(
%1+\frac{\left(1+\frac{c}{K_A}\right)^4}{\left(1+\frac{c}{K_A}\right)^4+e^{-\beta  \varepsilon }\left(1+\frac{c}{K_I}\right)^4}\frac{2[R]}{\K} \right)^{-1}.
%\end{equation}
Now suppose that the tetramerization region was removed within this model. Once
again, there would be twice as many dimers, each able to bind to DNA in only one
configuration, so that these two factors cancel each other. But now each dimer
must necessarily be active or inactive independently of all other dimers, and
therefore the probability of a repressor being active and the corresponding
equation fold-change would be given by \eref[AppendixEq1][AppendixEq2],
respectively. Upon changing the fourth powers to second powers, we expect that
the fold-change curves will dramatically shift.

Thus, by comparing fold-change measurements of a Lac repressor with and without
the tetramerization region, we can immediately distinguish between these two
models. \fref[figAllosteryModels1]\letter{A} shows fold-change data for the O2
operator. As in the main text, we use the RBS 1027 data set to determine the
thermodynamic parameters and then predict the remaining data sets with no
further fitting. After removing the tetramerization region, we would predict
that the fold-change curves shown in \fref[figAllosteryModels1]\letter{B} would
result. In contrast, if the independent dimer model is correct, we would expect
the fold-change curves to remain stationary when the tetramerization region is
removed. These plots demonstrate the need to use theoretical models not merely
to characterize known data, but to \textit{predict} the results of novel
experiments.

\begin{figure}[h]
	\centering \includegraphics[scale=\globalScalePlots]{SIfigure4.pdf}
	\caption{{\bf Effects of removing tetramerization region of Lac repressor.}
		\letterParen{A} Fold-change of tetrameric Lac repressor binding to the O2
		operator. Assuming that both repressor dimers are simultaneously active or
		inactive, \eref[AppendixEq4], the best fit parameters are $K_A = 45 \times
		10^{-6}\,\text{M}$, $K_I = 3 \times 10^{-6}\,\text{M}$, and $\beta \varepsilon
		= 4.5$. While these thermodynamic parameters are different from the
		corresponding values found from the independent dimer model,
		\eref[AppendixEq2], the resulting curves fit the data well. \letterParen{B}
		Once the tetramerization region is removed, the predicted fold-change is given
		by \eref[AppendixEq2] using these same thermodynamic parameters. The new
		fold-change curves are markedly different from the original curves. The opaque
		dots \textit{do not} represent data \talComment{not yet!}, but only server as a
		reference point for the theoretical curves. } \label{figAllosteryModels1}
\end{figure}

Another interesting aspect of removing the tetramerization region is that the
shift in the fold-change curves depends upon the $\varepsilon$ parameter.
\fref[figAllosteryModels2] shows the corresponding predictions assuming that
the repressor dimers are not independent and using the larger value $\beta
\varepsilon = 0$. In this case, the corresponding shift upon removing the
tetramerization is noticeably smaller. Hence, if a shift is observed in the
fold-change curves, the size of the shift would be an indirect measurement of
$\varepsilon$.

\begin{figure}[h]
	\centering \includegraphics[scale=\globalScalePlots]{SIfigure3.pdf}
	\caption{{\bf Dependence of tetramerization region removal on $\boldsymbol{\varepsilon}$.}
		\letterParen{A} As in \fref[figAllosteryModels1], we fit fold-change of Lac repressor with a tetramerization region binding to the O2 operator, but this time forcing $\beta \varepsilon = 0$. Using \eref[AppendixEq4], the best fit parameters are $K_A = 60 \times 10^{-6}\,\text{M}$, $K_I = 10 \times 10^{-6}\,\text{M}$, and $\beta \varepsilon = 0$. \letterParen{B} With this smaller $\beta \varepsilon$ value, the corresponding theoretical predictions exhibit a much larger fold-change than those in \fref[figAllosteryModels1]\letter{B}.
	}
	\label{figAllosteryModels2}
\end{figure}

%\pagebreak
%\phantom{...}
\pagebreak
%%%%%%%%%%%%%%%%%%%%%%%%%%%%%%%%%%%%%%%%%%%%%%%%%%%%%%%%%%%%%%%%%%%%%%%%%%%
\section{Parameter Degeneracy} \label{AppendixSloppiness}

\talComment{Can discuss this stuff below. Follow ion channels. First introduce the parameter degeneracy graph, then state that we are using $\beta \epsilon = -4.5$, then can discuss how the sloppiness approximation may arise. This latter is pasted below.}

Consider the probability that the allosteric Lac repressor is in the active state,
\begin{equation}\label{AppendixEq5}
p_A(c)=\frac{\left(1+\frac{c}{K_A}\right)^2}{\left(1+\frac{c}{K_A}\right)^2+e^{-\beta  \varepsilon }\left(1+\frac{c}{K_I}\right)^2}.
\end{equation}

It is hypothesized that in the absence of inducer ($c=0$), all of the repressors
are present in the active state, which implies $\beta \varepsilon \gg 1$. As
discussed in Appendix \ref{AppendixModel}, this assertion seems supported by the
available data, but must ultimately be validated by direct measurement, as is
possible by NMR \cite{Gardino2003, Boulton2016}. Given $\beta \varepsilon \gg
1$, we can approximate the term $e^{-\beta \varepsilon}
\left(1+\frac{c}{K_I}\right)^2 \approx e^{-\beta  \varepsilon
}\left(\frac{c}{K_I}\right)^2$ in the denominator of \eref[AppendixEq5]. \stephanieComment{I'm feeling that we may want to stick with the approximation from the Brewster/Franz data. Re-reading about the approximation in the context of the rest of our models makes me feel like it's a bit wishy-washy. \talComment{I think we should cut this out and just have all 3 parameters. We can discuss this approximation stuff in the SI.}} To see this,
note that both expressions are negligible compared to
$\left(1+\frac{c}{K_A}\right)^2$ for small $c$; on the other hand, the term
$e^{-\beta \varepsilon} \left(1+\frac{c}{K_I}\right)^2$ becomes non-negligible
once $e^{-\beta \varepsilon }\left(\frac{c}{K_I}\right)^2 \gtrsim 1$, in which
case $\frac{c}{K_I} \gg 1$ so that our approximation is again valid. Therefore,
we can approximate the probability that a repressor dimer is active as
\begin{equation}\label{AppendixEq6}
p_A(c) \approx \frac{\left(1+\frac{c}{K_A}\right)^2}{\left(1+\frac{c}{K_A}\right)^2+e^{-\beta  \varepsilon }\left(\frac{c}{K_I}\right)^2} \equiv \frac{\left(1+\frac{c}{K_A}\right)^2}{\left(1+\frac{c}{K_A}\right)^2+\left(\frac{c}{\KIeff}\right)^2},
\end{equation}
where we have introduced the effective dissociation constant $\KIeff = K_I
e^{\beta  \varepsilon/2}$. Substituting this result into \eref[eq5] yields the
complete formula
\begin{equation}\label{AppendixEq7}
\foldchange= \left(
1+\frac{\left(1+\frac{c}{K_A}\right)^2}{\left(1+\frac{c}{K_A}\right)^2+\left(\frac{c}{\KIeff}\right)^2}\frac{2 R}{N_{\text{NS}}}e^{-\beta \Delta\varepsilon_{RD,A}} \right)^{-1},
\end{equation}
%\begin{equation}\label{eq7}
%\foldchange= \left(
%1+\frac{\left(1+\frac{c}{K_A}\right)^2}{\left(1+\frac{c}{K_A}\right)^2+\left(\frac{c}{\KIeff}\right)^2}\frac{2[R]}{\K} \right)^{-1},
%\end{equation}
which predicts that given a concentration \(c\) of the inducer IPTG, \(R\)
copies of the Lac repressor, and the $N_{\text{NS}} = 4.6 \times 10^6$ \talComment{Do we want italics or non-italics subscripts? I don't care, but the figures should be consistent}
non-specific binding sites on the \textit{E. coli} genome, the fold-change in
gene expression will depend solely on 3 parameters: the DNA binding affinity of
the repressor ($\Delta\varepsilon_{RD,A}$) and the inducer binding affinities
for the repressor in the active state (\(K_A\)) and inactive state (\(\KIeff\)),
with this latter quantity also incorporating the difference in free energy
between the active and inactive states of the repressor.

%%%%%%%%%%%%%%%%%%%%%%%%%%%%%%%%%%%%%%%%%%%%%%%%%%%%%%%%%%%%%%%%%%%%%%%%%%%
\subsection{Background}

\talComment{The notation needs to be updated throughout the entire Appendix sections}

We begin with the fold-change for the Lac repressor,
\begin{equation}
\foldchange= \left(
1+\frac{\left(1+\frac{c}{K_A}\right)^2}{\left(1+\frac{c}{K_A}\right)^2+e^{-\beta \epsilon}\left(1 + \frac{c}{\KIeff}\right)^2}\frac{2 R}{N_{\text{NS}}}e^{-\beta \Delta\varepsilon_{RD,A}} \right)^{-1}.
\end{equation}
In the absence of inducer ($c = 0$), we can rewrite this equation as
\begin{equation}
\foldchange_{(c=0)}= \left(
1+\frac{1}{1+e^{-\beta \epsilon}}\frac{2 R}{N_{\text{NS}}}e^{-\beta \Delta\varepsilon_{RD,A}} \right)^{-1} = \left(
1+\frac{2 R}{N_{\text{NS}}}e^{-\beta \left( \Delta\tilde{\varepsilon}_{RD,A} \right) } \right)^{-1},
\end{equation}
where we have defined $\Delta\tilde{\varepsilon}_{RD,A} = \Delta\varepsilon_{RD,A} + \frac{1}{\beta} \log \left( 1 + e^{-\beta \epsilon} \right) $. Historically, the DNA binding energy $\Delta\varepsilon_{RD,A}$ was measured by fitting fold-change of multiple strains at $c = 0$, using measured values of $R$ (from quantitative Western blots) and $N_{\text{NS}} = 4.6
\times 10^6$ from the \textit{E. coli} genome size \cite{Garcia2011}. The value $\Delta\tilde{\varepsilon}_{RD,A}$ can be extracted from the data, but it was assumed that nearly all Lac repressors were active ($\beta \epsilon \gg 1$) so that $\Delta\tilde{\varepsilon}_{RD,A} \approx \Delta\varepsilon_{RD,A}$. In the following analysis, we will relax this assumption so that the full form of fold-change, incorporating all of the known parameter values, becomes
\begin{equation} \label{SIFullFoldChangeExpression}
\foldchange= \left(
1+\frac{\left(1+\frac{c}{K_A}\right)^2 \left( 1 + e^{-\beta \epsilon} \right)}{\left(1+\frac{c}{K_A}\right)^2+e^{-\beta \epsilon}\left(1 + \frac{c}{\KIeff}\right)^2}\frac{2 R}{N_{\text{NS}}}e^{-\beta \Delta\tilde{\varepsilon}_{RD,A}} \right)^{-1}.
\end{equation}
Using the measured values $\Delta\tilde{\varepsilon}_{RD,A} = -13.9 k_B T$ for
the O2 operator, $R=130$ for the RBS 1027 strain, and $N_{\text{NS}} = 4.6
\times 10^6$, we will analyze the degeneracy of this model in the following
sections.

%%%%%%%%%%%%%%%%%%%%%%%%%%%%%%%%%%%%%%%%%%%%%%%%%%%%%%%%%%%%%%%%%%%%%%%%%%%
\subsection{Degeneracy or Sloppiness in Parameter Values}

Non-linear regression (using least-squares or MCMC) is a standard tool to
extract parameters from data. However, care must be taken when
degenerate sets of parameters yield identical curves, since it may be impossible
to determine the correct choice of ``best-fit parameters'' from the spectrum of
potential candidates \cite{Transtrum2015}. This phenomenon, sometimes dubbed
``sloppiness,'' must be explicitly accounted for in a model before fitting.

\fref[SIfig5]\letter{A} demonstrates multiple fits with different parameter sets
to the RBS 1027 strain with the O2 promoter. As shown in
\fref[SIfig5]\letter{B}, given any $\beta \epsilon$ both $K_A$ and $K_I$ can be
fit in such a way so that the resulting fold-change expression closely matches
the data. We point out some key features of this result: (1) the coefficient of
determination $R^2 = 0.99$ is essentially the same for each fit, indicating that
each fit closely matches the data; (2) the value of $K_A$ is nearly flat,
although it is slightly different for positive ($K_A = 185\,\,\mu\text{M}$) and
negative ($K_A = 145\,\,\mu\text{M}$) values of $\beta \epsilon$; (3) the value
of $K_I$ is constant ($K_I = 6\,\,\mu\text{M}$) for negative $\beta \epsilon$
and varies logarithmically ($K_I \approx 5 e^{-\beta \epsilon/2}$) for positive
$\beta \epsilon$; and (4) it can be shown that after $\beta \epsilon$ is fixed,
the resulting model is no longer sloppy, so that any values other than the $K_A$
and $K_I$ shown in \fref[SIfig5]\letter{B} will result in worse fits.

\begin{figure}[h]
	\centering \includegraphics{SIfigure5.pdf} \caption{{\bf Degeneracy in best fit
			parameters of fold-change.} \letterParen{A} Fitting fold-change of RBS 1027 on
		O2 operator ($n=10$ measurements for each IPTG concentration) using 10 data
		sets of I forced $\beta \varepsilon_R$ to take on a particular value and fit
		$K_A$ and $K_I$ to the data using \eref[SIFullFoldChangeExpression] using
		different values of $\beta \epsilon \in [-10,10]$. The resulting best-fit
		curves are nearly identical regardless of the $\beta \epsilon$ value.
		\letterParen{B} The corresponding values of $\log\left( K_A \right)$ and $\log
		\left( K_I \right)$ for each value of $\beta \epsilon$. For every $\beta
		\epsilon$ value, the coefficient of determination $R^2$ (dashed line) is
		approximately 1, indicating an excellent fit. In the two regimes $\beta
		\epsilon < 0$ and $\beta \epsilon > 0$, the relationship between $\beta
		\epsilon$ and $\log\left( K_A \right)$ and $\log \left( K_I \right)$ is nearly
		linear. \talComment{Verify that $R^2$ is residuals or standardized residuals.}}
	\label{SIfig5}
\end{figure}

Without further input, \fref[SIfig5]\letter{B} shows the total amount of
information that can be extracted about the parameters $K_A$, $K_I$, and
$\epsilon$ from the data. In other words, these parameters cannot be
individually fit, instead only the relations $K_A(\epsilon)$ and $K_I(\epsilon)$
can be learned. Another experiment would have to be conducted to independently
measure $\epsilon$ and thereby fix $K_A$ and $K_I$. Such an experiment would be
possible, for example, by using NMR to directly measure the fraction
$1/(1+e^{-\beta\epsilon})$ of active repressors in the absence of IPTG
\cite{Gardino2003, Boulton2016}.

To make further analytic progress, we will assume that $e^{- \beta \epsilon} \ll
1$. \manuelComment{Can we use Stephanie's insight to get the $e^{-\beta \epsilon} \ll  1$ assumption? \talComment{I don't think so. Not only does that ignore her new insight on the operator copy number mattering, but I also don't know how definitive the 1st version analysis was (i.e. I know that $\beta \epsilon = 4.5$ emerged out of it, but I don't know how much better that is than lower $\beta \epsilon$ values.)}} This assumption must ultimately be validated by a measurement of $\beta
\epsilon$, but we note since wild type \textit{E. coli} has $R=11$ repressors
total, negative $\beta \epsilon$ would imply that at most five repressors are
active in the absence of IPTG.

In the regime $e^{- \beta \epsilon} \ll
1$, we find from \fref[SIfig5]\letter{B} that $K_A = 145\,\,\mu\text{M}$ and $K_I e^{\beta \epsilon/2} \approx 5$. We now show how the parameter combination $K_I e^{\beta \epsilon/2}$ emerges from the fold-change equation. Consider the fractional part of \eref[SIFullFoldChangeExpression],
\begin{align}
\frac{\left(1+\frac{c}{K_A}\right)^2 \left( 1 + e^{-\beta \epsilon} \right)}{\left(1+\frac{c}{K_A}\right)^2+e^{-\beta \epsilon}\left(1 + \frac{c}{K_I}\right)^2}
&\approx
\frac{\left(1+\frac{c}{K_A}\right)^2}{\left(1+\frac{c}{K_A}\right)^2+\left(e^{-\beta \epsilon/2} + e^{-\beta \epsilon/2}\frac{c}{K_I}\right)^2} \label{eqSIstep2}\\
&\approx
\frac{\left(1+\frac{c}{K_A}\right)^2}{\left(1+\frac{c}{K_A}\right)^2+\left( e^{-\beta \epsilon/2}\frac{c}{K_I}\right)^2}.\label{eqSIstep3}
\end{align}
In the first step we approximated the numerator as $1 + e^{-\beta \epsilon}
\approx 1$. In the second step, we note that $\left(1+\frac{c}{K_A}\right)^2 \ge
1$ so that adding the small $e^{-\beta \epsilon/2}$ term to $e^{-\beta
	\epsilon/2}\frac{c}{K_I}$ before squaring will make little difference. This last
assumption will be rigorously analyzed in the following section, where we
quantify the error in going from \eref[eqSIstep2] to \eref[eqSIstep3]. For now,
we note that inserting the fractional form \eref[eqSIstep3] back into the
fold-change equation \eref[SIFullFoldChangeExpression] yields,
\begin{equation} \label{SIFullFoldChangeExpressionSloppiless}
\foldchange = \left(
1+\frac{\left(1+\frac{c}{K_A}\right)^2}{\left(1+\frac{c}{K_A}\right)^2+\left(\frac{c}{\KIeff}\right)^2}\frac{2 R}{N_{\text{NS}}}e^{-\beta \Delta\tilde{\varepsilon}_{RD,A}} \right)^{-1},
\end{equation}
where we have introduced the new parameter $\KIeff = K_I e^{\beta \epsilon/2}$. In this form, it is clear that only $K_A$ and the parameter combination $\KIeff$ can be extracted from fold-change data.


%%%%%%%%%%%%%%%%%%%%%%%%%%%%%%%%%%%%%%%%%%%%%%%%%%%%%%%%%%%%%%%%%%%%%%%%%%%
\subsection{Approximate versus Exact Form of Fold-Change}

In this section, we determine the error of approximating \eref[eqSIstep2] as
\eref[eqSIstep3] in the limit $e^{- \beta \epsilon} \ll 1$. As a point of
reference, the two functions are shown in \fref[SIfig6]\letter{A} using the
values $\beta \epsilon = 5$, $K_A = 140\,\,\mu\text{M}$, and $K_I =
0.4\,\,\mu\text{M}$. \fref[SIfig6]\letter{B} demonstrates how this discrepancy
exponentially decays for larger $\beta \epsilon$ values.

\begin{figure}[h]
	\centering \includegraphics{SIfigure6.pdf} \caption{{\bf Exact versus
			approximate expressions within fold-change.} \letterParen{A} Difference between
		the exact form \eref[eqSIstep2] and the approximation \eref[eqSIstep3] using
		the parameters $\beta \epsilon = 5$, $K_A = 140\,\,\mu\text{M}$, and $K_I =
		0.4\,\,\mu\text{M}$. \letterParen{B} Difference between the exact and
		approximate forms for a range of $\beta \epsilon$ values, using the best fit
		$K_A$ and $K_I$ from \fref[SIfig5]\letter{B} in each case. The curve is
		proportional to $e^{-\beta \epsilon/2}$ and shows that this difference
		diminishes exponentially with increasing $\beta \epsilon$.} \label{SIfig6}
\end{figure}

We first find the maximum difference between these two functions, which occurs
around $c = 3\,\,\mu\text{M}$ in \fref[SIfig6]. This is accomplished by first
taking the derivative of the difference and finding the concentration $c^*$ at
which the derivative will equal 0. By Taylor expanding the derivative about
$e^{- \beta \epsilon} \ll 1$ to first order before solving for the root, we
obtain
\begin{equation}
c^* \approx \frac{K_I}{2 K_A} \left( \sqrt{1+4\frac{1-\frac{K_I}{K_A}}{\left( \frac{K_I}{K_A} \right)^2 + 2 e^{- \beta \epsilon}}} - 1 \right).
\end{equation}
We now substitute $c=c^*$ to obtain the difference between the exact
and approximate expressions, \eref[eqSIstep2][eqSIstep3], taking another Taylor
series about $e^{- \beta \epsilon} \ll 1$ to obtain the maximum difference in
our approximation,
\begin{equation}
\text{max difference} \approx \frac{e^{- \beta \epsilon}}{2 \frac{K_I}{K_A} - \left( \frac{K_I}{K_A} \right)^2}.
\end{equation}
This formula greatly overstates the maximal discrepancy, predicting a value of
1.1 for the parameter values in \fref[SIfig6]\letter{A} even though the actual
maximal difference in the plot is 0.05. However, this formula allows us to
understand how the maximum difference scales with $e^{- \beta \epsilon}$. From
\fref[SIfig5]\letter{B}, we see that as $\epsilon$ increases, $K_A$ stays
constant while $K_I \propto e^{-\beta \epsilon/2}$, which suggests that the
maximal difference should shrink as $e^{-\beta \epsilon/2}$, as is indeed seen
in \fref[SIfig6]\letter{B}.

%\pagebreak
%\phantom{...}
\pagebreak
%%%%%%%%%%%%%%%%%%%%%%%%%%%%%%%%%%%%%%%%%%%%%%%%%%%%%%%%%%%%%%%%%%%%%%%%%%%
\section{Bayesian Parameter Estimation} \label{AppendixParamEstimation}

\begin{figure}[h]
	\centering \includegraphics[scale=0.5]{SI_20160923_fitcompare_summaryO1.pdf}
	\caption{{\bf Parameter Estimation comparison for O1 strains.} Fold change in expression is plotted as a function of IPTG concentration for all strains containing an O1 operator. The solid points correspond to our experimental data, where error bars in fold change measurements refering to the SEM (n=10). The solid lines correspond to \eref[eq7] using the parameter estimates of $K_I$, $K_A$, and $\epsilon$. Each row uses a single set of parameter values based on the strain noted on the left axis. THe shaded plots along the diagonal are those where the parameter estimates are ploted along with the data used to infer them. Values for repressor copy number and operator binding energy are from \cite{Garcia2011}.  The shaded region on the curve represents the uncertainty from our parameter estimates and reflect the 95\%highest probability density region of the parameter predictions for $K_I$, $K_A$, and $\epsilon$.}
	\label{SIfig7}
\end{figure}

\begin{figure}[h]
	\centering \includegraphics[scale=0.5]{SI_20160923_fitcompare_summaryO2.pdf}
	\caption{{\bf Parameter Estimation comparison for O2 strains.} Fold change in expression is plotted as a function of IPTG concentration for all strains containing an O2 operator. The solid points correspond to our experimental data, where error bars in fold change measurements refering to the SEM (n=10). The solid lines correspond to \eref[eq7] using the parameter estimates of $K_I$, $K_A$, and $\epsilon$. Each row uses a single set of parameter values based on the strain noted on the left axis. THe shaded plots along the diagonal are those where the parameter estimates are ploted along with the data used to infer them. Values for repressor copy number and operator binding energy are from \cite{Garcia2011}.  The shaded region on the curve represents the uncertainty from our parameter estimates.}
	\label{SIfig8}
\end{figure}

\begin{figure}[h]
	\centering \includegraphics[scale=0.5]{SI_20160923_fitcompare_summaryO3.pdf}
	\caption{{\bf Parameter Estimation comparison for O3 strains.} Fold change in expression is plotted as a function of IPTG concentration for all strains containing an O3 operator. The solid points correspond to our experimental data, where error bars in fold change measurements refering to the SEM (n=10). The solid lines correspond to \eref[eq7] using the parameter estimates of $K_I$, $K_A$, and $\epsilon$. Each row uses a single set of parameter values based on the strain noted on the left axis. THe shaded plots along the diagonal are those where the parameter estimates are ploted along with the data used to infer them. Values for repressor copy number and operator binding energy are from \cite{Garcia2011}.  The shaded region on the curve represents the uncertainty from our parameter estimates.}
	\label{SIfig9}
\end{figure}

\begin{figure}[h]
	\centering \includegraphics[scale=0.5]{SI_20160923_fitcompare_summaryOid.pdf}
	\caption{{\bf Parameter Estimation comparison for Oid strains.} Fold change in expression is plotted as a function of IPTG concentration for all strains containing an Oid operator. The solid points correspond to our experimental data, where error bars in fold change measurements refering to the SEM (n=10). The solid lines correspond to \eref[eq7] using the parameter estimates of $K_I$, $K_A$, and $\epsilon$. Each row uses a single set of parameter values based on the strain noted on the left axis. THe shaded plots along the diagonal are those where the parameter estimates are ploted along with the data used to infer them. Values for repressor copy number and operator binding energy are from \cite{Garcia2011}.  The shaded region on the curve represents the uncertainty from our parameter estimates.}
	\label{SIfig10}
\end{figure}


%%%%%%%%%%%%%%%%%%%%%%%%%%%%%%%%%%%%%%%%%%%%%%%%%%%%%%%%%%%%%%%%%%%%%%%%%%%%
%\section{Old Sloppiness Appendix Below - Can now be removed}
%
%\talComment{An initial discussion of sloppiness and the $\varepsilon$ parameter. Needs to be updated with one of our actual data sets. Perhaps if we have enough data sets this sloppiness will go away?}
%
%We have all seen by now that we can wiggle the best fit parameters around and still get good fitting. To better visualize the problem, I took Stephanie's 2016-06-18 data for HG 104 and RBS 1027 and fit it to the form \eref[AppendixSloppinessEq1], forcing the $\beta \varepsilon_R$ to vary between -10 and 10. \fref[SIfig1] shows the results.
%\begin{figure}[h]
%	\centering \includegraphics{SIfigure1.pdf}
%	\caption{{\bf Many best fit parameters generate good fits.} \letterParen{A} I forced $\beta \varepsilon_R$ to take on a particular value and fit $K_A$ and $K_I$ to the data using \eref[AppendixSloppinessEq1]. All of the resulting fit curves look good. \letterParen{B} The best fit $K_A$ and $K_I$ values. This graph seemed quite provacative to me, since the values of $K_A$ and $K_I$ seem to be linear curves on this logarithmic plot.}
%	\label{SIfig1}
%\end{figure}
%
%Rather than continuing to beat around the bush, I thought it was time to tackle this issue head on. The following analysis is for the following $n=2$
%exponent form,
%\begin{align} \label{AppendixSloppinessEq1}
%\foldchange &= \frac{1}{1+\frac{\left( 1 + \frac{c}{K_A} \right)^2 \left(1 + e^{-\beta \varepsilon_R} \right)}{\left( 1 + \frac{c}{K_A} \right)^2 + e^{-\beta \varepsilon_R} \left( 1 + \frac{c}{K_I} \right)^2} \frac{2 \Rtot}{N_{NS}} e^{-\beta \Delta\varepsilon_{\text{O2}}^{(\text{Hernan})}}}\nonumber\\
%&\equiv \frac{1}{1+\frac{\left( 1 + \frac{c}{K_A} \right)^2 \left(1 + e^{-\beta \varepsilon_R} \right)}{\left( 1 + \frac{c}{K_A} \right)^2 + e^{-\beta \varepsilon_R} \left( 1 + \frac{c}{K_I} \right)^2} r},
%\end{align}
%where for notational convenience I have defined $r = \frac{2 \Rtot}{N_{NS}}
%e^{-\beta \Delta\varepsilon_{\text{O2}}^{(\text{Hernan})}}$ (with $r \approx 1$ for
%HG 104, using Hernan's values). Recall that $K_I < K_A$, since the inducer makes
%the repressor more likely to assume the inactive state. I will analyze this
%fold-change equation in the two limits: $e^{-\beta \varepsilon_R} \ll 1$ and
%$e^{-\beta \varepsilon_R} \gg 1$.
%
%\subsection{$\boldsymbol{e^{-\beta \varepsilon_R} \ll 1}$}
%
%In the limit $e^{-\beta \varepsilon_R} \ll 1$, the fold-change equation becomes
%\begin{align} \label{AppendixSloppinessEq2}
%\foldchange &= \frac{1}{1+\frac{\left( 1 + \frac{c}{K_A} \right)^2}{\left( 1 + \frac{c}{K_A} \right)^2 + e^{-\beta \varepsilon_R} \left(\frac{c}{K_I} \right)^2} r} \nonumber\\
%&\equiv \frac{1}{1+\frac{\left( 1 + \frac{c}{K_A} \right)^2}{\left( 1 + \frac{c}{K_A} \right)^2 + \left(\frac{c}{\widetilde{K}_I} \right)^2} r}.
%\end{align}
%where we have defined $\widetilde{K}_I = K_I e^{\beta \varepsilon_R/2}$. This
%suggests that what we are actually fitting $K_A$ and $\widetilde{K}_I$ in this
%regime, as confirmed by \fref[SIfig2] in the region $e^{-\beta \varepsilon_R} \ll 1$ (note that the $K_I e^{\beta \varepsilon_R/2}$ values do not change).
%%where we have made two approximations. First, we assumed $\left(1 + e^{-\beta
%% \varepsilon_R} \right) \approx 1$ in the numerator. Second, we used $\left( 1 +
%% \frac{c}{K_A} \right)^2 + e^{-\beta \varepsilon_R} \left( 1 + \frac{c}{K_I}
%% \right)^2 \approx \left( 1 + \frac{c}{K_A} \right)^2 + e^{-\beta \varepsilon_R}
%% \left(\frac{c}{K_I} \right)^2$ in the denominator because the $e^{-\beta
%% \varepsilon_R} \left(1 + \frac{c}{K_I} \right)^2$ term
%\begin{figure}[h]
%	\centering \includegraphics{SIfigure2.pdf} \caption{{\bf The altered
%			form of $K_I$ we are actually fitting when $\boldsymbol{e^{-\beta \varepsilon_R}
%				\ll 1}$.} Redrawing of \fref[SIfig1]\letter{B} showing $K_I$ and $\widetilde{K}_I
%		= K_I e^{\beta \varepsilon_R/2}$, the latter forming a straight line when
%		$e^{-\beta \varepsilon_R} \ll 1$.} \label{SIfig2}
%\end{figure}
%
%With only two effective parameters, you should be able to figure out their
%values without doing any fitting. To do this, consider three important
%characteristics of a titration curve: (1) the \textit{leakiness} equals the fold-change
%when $c=0$; (2) the \textit{dynamic range} equals the difference between the maximal
%fold-change at $c\to\infty$ and $c=0$; and $[EC_{50}]$ equals the concentration at
%which fold-change reaches halfway between its minimum and maximum values.
%
%Using \eref[AppendixSloppinessEq1],
%\begin{align}
%\lim_{c = 0}\foldchange &= \frac{1}{1+r}\\
%\lim_{c \to \infty}\foldchange &= \frac{\left(\frac{K_A}{\widetilde{K}_I}\right)^2 + 1}{\left(\frac{K_A}{\widetilde{K}_I}\right)^2 + 1+r}
%\end{align}
%so that
%\begin{align}
%\text{leakiness} &= \frac{1}{1+r} \label{AppendixSloppinessEq3} \\
%\text{dynamic range} &= \frac{\left(\frac{K_A}{\widetilde{K}_I}\right)^2 + 1}{\left(\frac{K_A}{\widetilde{K}_I}\right)^2 + 1+r} - \frac{1}{1+r}. \label{AppendixSloppinessEq4}
%\end{align}
%Since you can read the leakiness and dynamic range straight off the data, \eref[AppendixSloppinessEq4] already gives us one relationship for $K_A$ and $\widetilde{K}_I$,
%\begin{equation} \label{AppendixSloppinessEq5}
%\frac{K_A}{\widetilde{K}_I} = \left( \left(1+r\right) \frac{\text{dynamic range} - \frac{1}{1+r}}{1-\text{dynamic range}} \right)^{1/2}.
%\end{equation}
%
%We next solve for $[EC_{50}]$, showing first an exact formula and then an approximation that can be achieved by noting that $\widetilde{K}_I < K_A$ and that $r \approx 1$,
%\begin{align}
%[EC_{50}] &= \frac{K_A \widetilde{K}_I^2 \left(1+r\right) + \sqrt{K_A^2 \widetilde{K}_I^2 \left(1+r\right) \left(K_A^2 + 2 \widetilde{K}_I^2 \left(1+r\right) \right)}}{K_A^2 + \widetilde{K}_I^2 \left(1+r\right)}\nonumber\\
%&\approx \widetilde{K}_I \left(1+r\right)^{1/2}.
%\end{align}
%Combining this formula with \eref[AppendixSloppinessEq5], we can directly find the value of our two parameters without resorting to fitting,
%\begin{align}
%\widetilde{K}_I &= \frac{[EC_{50}]}{\left(1+r\right)^{1/2}}\\
%K_A &= [EC_{50}] \left(\frac{\text{dynamic range} - \frac{1}{1+r}}{1-\text{dynamic range}} \right)^{1/2}.
%\end{align}
%
%
%\subsection{$\boldsymbol{e^{-\beta \varepsilon_R} \gg 1}$}
%
%In this limit, \eref[AppendixSloppinessEq1] becomes
%\begin{equation}
%\foldchange = \frac{1}{1+\left(\frac{1 + \frac{c}{K_A}}{1 + \frac{c}{K_I}}\right)^2 r},
%\end{equation}
%since in the numerator $\left( 1 + e^{-\beta \varepsilon_R} \right) \approx
%e^{-\beta \varepsilon_R}$ while in the denominator the
%$\left(1+\frac{c}{K_A}\right)^2$ term is always smaller than the $e^{-\beta
%	\varepsilon_R} \left(1+\frac{c}{K_I}\right)^2$ term. In this case, we are also
%fitting only two parameters (only $K_A$ and $K_I$), and we see that the $\beta
%\varepsilon_R$ value should not influence these parameters, as is confirmed in
%\fref[SIfig1]\letter{B}.
%
%You can play the same game as in the previous section to get the values of these two parameters from the dynamic range and $[EC_{50}]$. For now, I skip the details and just write the results,
%\begin{align}
%K_I &= \frac{[EC_{50}]}{\left(2+r\right)^{1/2}-1}\\
%K_A &= [EC_{50}] \frac{r^{1/2}}{\left(2+r\right)^{1/2}-1} \left(\frac{\text{dynamic range}}{1-\text{dynamic range}}\right).
%\end{align}
%
%\pagebreak
%%%%%%%%%%%%%%%%%%%%%%%%%%%%%%%%%%%%%%%%%%%%%%%%%%%%%%%%%%%%%%%%%%%%%%%%%%%%
%\subsection{A Detailed Explanation of the Sloppiness Approximation}
%
%In this section, we show that the approximation
%\begin{equation}
%p_A(c)=\frac{\left(1+\frac{c}{K_A}\right)^2}{\left(1+\frac{c}{K_A}\right)^2+e^{-\beta \epsilon }\left(1+\frac{c}{K_I}\right)^2}\approx \frac{\left(1+\frac{c}{K_A}\right)^2}{\left(1+\frac{c}{K_A}\right)^2+e^{-\beta
%		\epsilon }\left(\frac{c}{K_I}\right)^2}
%\end{equation}
%is valid provided \(\beta \epsilon \gg 1\). In terms of the dimensionless
%parameters \(\tilde{c}=\frac{c}{K_A}\) and \(\tilde{K}=\frac{K_I}{K_A}\), the
%exact and approximate forms become
%\begin{equation} \label{AppendixEqSloppiness}
%p_A\left(\tilde{c}\right)=\frac{\left(1+\tilde{c}\right)^2}{\left(1+\tilde{c}\right)^2+e^{-\beta \epsilon }\left(1+\frac{\tilde{c}}{\tilde{K}}\right)^2}\approx
%\frac{\left(1+\tilde{c}\right)^2}{\left(1+\tilde{c}\right)^2+e^{-\beta \epsilon }\left(\frac{\tilde{c}}{\tilde{K}}\right)^2}.
%\end{equation}
%
%To give a sense of the parameters, our standard fitting methods predicted
%\(K_A=45 \times 10^{-6}\,\text{M}\), \(K_I=3 \times 10^{-6}\,\text{M}\), and
%\(\beta \epsilon =4.5\) using data in the range \(c\in
%\left[10^{-7},10^{-2}\right]\text{M}\). This corresponds to the dimensionless parameter
%\(\tilde{K}=\frac{1}{15}\) and the dimensionless range of concentrations \(\tilde{c}\in \left[10^{-3},10^3\right]\). \fref[SIfigSloppiness] shows a plot of both forms of
%$p_A(\tilde{c})$ from \eref[AppendixEqSloppiness] using these values,
%demonstrating visually that the approximation is valid.
%
%\begin{figure}[h]
%	\centering \includegraphics{sloppinessExplanation.pdf} \caption{{\bf An
%			approximation of the probability \(\boldsymbol{p_A\left(\tilde{c}\right)}\) that the Lac
%			repressor will be active.} Using \(\beta \epsilon =4.5\) and
%		\(\tilde{K}=\frac{1}{15}\), the approximate form in \eref[AppendixEqSloppiness]
%		is nearly identical to the exact form.} \label{SIfigSloppiness}
%\end{figure}
%
%To demonstrate the validity of this approximation analytically, we first rewrite \(p_A\left(\tilde{c}\right)\) as
%\begin{equation}
%p_A\left(\tilde{c}\right)=\frac{1}{1+e^{-\beta \epsilon }\frac{\left(1+\frac{\tilde{c}}{\tilde{K}}\right)^2}{\left(1+\tilde{c}\right)^2}}
%\end{equation}
%and then expand each term in the numerator,
%\begin{equation}
%p_A\left(\tilde{c}\right)=\frac{1}{1+e^{-\beta \epsilon }\frac{1}{\left(1+\tilde{c}\right)^2}+2e^{-\beta \epsilon }\frac{\frac{\tilde{c}}{\tilde{K}}}{\left(1+\tilde{c}\right)^2}+e^{-\beta
%		\epsilon }\frac{\left(\frac{\tilde{c}}{\tilde{K}}\right)^2}{\left(1+\tilde{c}\right)^2}}.
%\end{equation}
%
%First, note that the term \(e^{-\beta \epsilon }\frac{1}{\left(1+\tilde{c}\right)^2}\) satisfies
%\begin{equation}
%e^{-\beta \epsilon }\frac{1}{\left(1+\tilde{c}\right)^2}<e^{-\beta \epsilon }\ll 1
%\end{equation}
%and can therefore be dropped from the denominator since it is overwhelmed by the
%\(1\) term preceding it. Next, consider the term \(2e^{-\beta \epsilon
%}\frac{\frac{\tilde{c}}{\tilde{K}}}{\left(1+\tilde{c}\right)^2}\). For small
%\(\tilde{c}\) (when $e^{-\beta \epsilon }\frac{\tilde{c}}{\tilde{K}}\lesssim 1$),
%this term is dominated by the 1 term. The opposite limit of large $\tilde{c}$
%(when \(e^{-\beta \epsilon }\frac{\tilde{c}}{\tilde{K}}\gtrsim 1\)), necessarily
%implies \(\frac{\tilde{c}}{\tilde{K}}\gg 1\) by virtue of \(e^{-\beta \epsilon }\ll
%1\). But if
%\(\frac{\tilde{c}}{\tilde{K}}\gg 1\), then the \(2e^{-\beta \epsilon
%}\frac{\frac{\tilde{c}}{\tilde{K}}}{\left(1+\tilde{c}\right)^2}\) term is
%dominated by the \(e^{-\beta \epsilon
%}\frac{\left(\frac{\tilde{c}}{\tilde{K}}\right)^2}{\left(1+\tilde{c}\right)^2}\)
%term. In other words, regardless of the value of \(\tilde{c}\), this term can be
%neglected due to one of the other terms,
%\begin{equation}
%\begin{cases}
%2e^{-\beta \epsilon }\frac{\frac{\tilde{c}}{\tilde{K}}}{\left(1+\tilde{c}\right)^2}\ll 1 & e^{-\beta \epsilon }\frac{\tilde{c}}{\tilde{K}}\lesssim 1 \\
%2e^{-\beta \epsilon }\frac{\frac{\tilde{c}}{\tilde{K}}}{\left(1+\tilde{c}\right)^2}\ll e^{-\beta \epsilon }\frac{\left(\frac{\tilde{c}}{\tilde{K}}\right)^2}{\left(1+\tilde{c}\right)^2} & e^{-\beta \epsilon }\frac{\tilde{c}}{\tilde{K}}\gtrsim 1.
%\end{cases}
%\end{equation}
%After dropping this term, we arrive at the desired form of \(p_A\left(\tilde{c}\right)\), namely,
%\begin{equation}
%p_A\left(\tilde{c}\right)=\frac{1}{1+e^{-\beta \epsilon }\frac{\left(\frac{\tilde{c}}{\tilde{K}}\right)^2}{\left(1+\tilde{c}\right)^2}}=\frac{\left(1+\tilde{c}\right)^2}{\left(1+\tilde{c}\right)^2+e^{-\beta
%		\epsilon }\left(\frac{\tilde{c}}{\tilde{K}}\right)^2}.
%\end{equation}
%
%Of course, the real benefit of this form is that it permits us to absorb \(\beta \epsilon\) into \(\tilde{K}\), thereby removing all sloppiness from
%our model!

\pagebreak
