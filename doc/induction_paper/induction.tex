% Template for PLoS
% Version 3.0 December 2014

\documentclass[10pt,letterpaper]{article}
\usepackage[top=0.85in,left=1.00in,footskip=0.75in]{geometry}

% Use adjustwidth environment to exceed column width (see example table in text)
\usepackage{changepage}

% Use Unicode characters when possible
\usepackage[utf8]{inputenc}

% textcomp package and marvosym package for additional characters
\usepackage{textcomp,marvosym}

% fixltx2e package for \textsubscript
\usepackage{fixltx2e}

% amsmath and amssymb packages, useful for mathematical formulas and symbols
\usepackage{amsmath,amssymb}

% cite package, to clean up citations in the main text. Do not remove.
\usepackage{cite}

% Use nameref to cite supporting information files (see Supporting Information section for more info)
\usepackage{nameref,hyperref}

% line numbers
\usepackage[right]{lineno}

% ligatures disabled
\usepackage{microtype}
\DisableLigatures[f]{encoding = *, family = * }

% rotating package for sideways tables
\usepackage{rotating}

% Remove comment for double spacing
%\usepackage{setspace} 
%\doublespacing

% Text layout
\raggedright
\setlength{\parindent}{0.5cm}
%\textwidth 5.25in 
\textheight 8.75in

% Bold the 'Figure #' in the caption and separate it from the title/caption with a period
% Captions will be left justified
\usepackage[aboveskip=1pt,labelfont=bf,labelsep=period,justification=raggedright,singlelinecheck=off]{caption}

% Use the PLoS provided BiBTeX style
\bibliographystyle{plos2009}

% Remove brackets from numbering in List of References
\makeatletter
\renewcommand{\@biblabel}[1]{\quad#1.}
\makeatother

% Leave date blank
\date{}

% Header and Footer with logo
\usepackage{lastpage,fancyhdr,graphicx}
\pagestyle{myheadings}
\pagestyle{fancy}
\fancyhf{}
%\lhead{\includegraphics[natwidth=1.3in,natheight=0.4in]{PLOSlogo.png}}
\rfoot{\thepage/\pageref{LastPage}}
\renewcommand{\footrule}{\hrule height 2pt \vspace{2mm}}
%\fancyheadoffset[L]{1.00in}
%\fancyfootoffset[L]{1.00in}
%\lfoot{\sf PLOS}

%%%%%%%%%%%%%%%%%%%%%%%%%%%%%%%%%%%%%%%%%%%%%%%%%%%%%%%%
%%%%%%%%%%%% OPTIONAL MACRO DEFINITIONS %%%%%%%%%%%%%%%%
%%%%%%%%%%%%%%%%%%%%%%%%%%%%%%%%%%%%%%%%%%%%%%%%%%%%%%%%

\newcommand\globalScalePlots{1}

% For commenting
\usepackage{color} % ULTIMATELY delete since I can't have colored text???\
\usepackage[dvipsnames]{xcolor}
\newcommand{\griffinComment}[1]{\textcolor{CadetBlue}{(GC:~#1)}}
\newcommand{\manuelComment}[1]{\textcolor{ForestGreen}{(MR:~#1)}}
\newcommand{\nathanComment}[1]{\textcolor{Emerald}{(MR:~#1)}}
\newcommand{\stephanieComment}[1]{\textcolor{BrickRed}{(SB:~#1)}}
\newcommand{\talComment}[1]{\textcolor{Plum}{(TE:~#1)}}
\newcommand{\robComment}[1]{\textcolor{Orange}{(RP:~#1)}}

% To define more useful LaTeX commands
\usepackage{xparse}

% Equation referencing
\DeclareDocumentCommand \eref{oooo} {\IfNoValueTF{#2}{Eq.~(\ref{#1})}{\IfNoValueTF{#3}{Eqs.~(\ref{#1}) and (\ref{#2})}{\IfNoValueTF{#4}{Eqs.~(\ref{#1})-(\ref{#3})}{Eqs.~(\ref{#1})-(\ref{#4})}}}}

\newcommand{\bareEq}[1]{(\ref{#1})} % Equations

% Figure referencing
\DeclareDocumentCommand \fref{ooo} {\IfNoValueTF{#2}{Fig.~\ref{#1}}{\IfNoValueTF{#3}{Figs.~\ref{#1} and \ref{#2}}{Figs.~\ref{#1}-\ref{#3}}}}

% Letters for figure sub-parts
\newcommand{\letter}[1]{#1} % For main text. Ex: As shown in Fig. 12A
\newcommand{\letterParen}[1]{(#1)} % For captions. Ex: (A) Low limit and (B) high limit

% Shortcuts for often-used symbols that may change notation
\newcommand{\K}{K_{\text{DNA}}}
\newcommand \foldchange{\operatorname{fold-change}}
\newcommand{\Rtot}{R_{\text{Tot}}}

%% END MACROS SECTION


\begin{document}

% import macro for the figures path
% this will not be synchronized to the GitHub repository since each collaborator will adapt it to 
%his/her own directory structure
\graphicspath{{C:/Users/Tal/Dropbox/Research/mwc_induction/figures/}}
	
\vspace*{0.35in}

% Title must be 150 characters or less
\begin{flushleft}
{\Large
\textbf\newline{Thermodynamics of Titration Induction}
}
%\newline
%% Insert Author names, affiliations and corresponding author email.
%\\
%author 1
%\\
%author 2
%\\
%* phillips@pboc.caltech.edu
\end{flushleft}


% Please keep the abstract below 300 words
\section*{Abstract} 

Processes such as simple repression have been nailed down. Now we can build upon these simple models to understand more complicated interactions such as induction, which serve as the cornerstone for cell sensing and signaling.

Role of prediction in biology, not just to characterize data but to predict future experiments.

Use the Monod-Wyman-Changeux model of
allostery to understand the data within a single, unified framework.


%%%%%%%%%%%%%%%%%%%%%%%%%%%%%%%%%%%%%%%%%%%%%%%%%%%%%%%%%%%%%%%%%%%%%%%%%%%
\section*{eLife digest} 

\textit{Simple explanation of project}

%\linenumbers

\section*{Introduction}

The lactose (\textit{lac}) system, responsible for the metabolism of lactose in
\textit{Escherichia coli}, is arguably one of the best studied genes of all time
\cite{MullerHill1996a}. Due to its central importance in the areas of biology
ranging from transcription to evolution to allostery \cite{Razo-Mejia2014}
\talComment{More references!}, a vast amount of data has been collected for this
system which has given rise to the many quantitative models.

The Lac repressor, the key molecule which binds to the Lac promoter and inhibits
gene expression, was isolated 50 years ago \cite{Gilbert1966}. It was quickly
understood that the Lac repressor was allosteric, existing in two different
conformations with different DNA binding affinities, with the transition between
these two states mediated by an effector site separated by \_\_\_
\talComment{Distance in amino acids and in 3D shortest-line distance?} from the
DNA binding domai \talComment{cite the crap out of everything here}. This two
state system was amenable to the Monod-Wyman-Changeux (MWC) model of allostery
\cite{MONOD1965}.

Since then, the Lac repressor's ability to inhibit gene expression has been carefully quantified
as a function of repressor copy number and promoter affinity \cite{Garcia2011}
as well as gene copy number and competitor binding sites \cite{Weinert2014}. All
of these tunable parameters have been well characterized by statistical
mechanical models grounded upon the MWC framework. But importantly, such models
have allowed us to go beyond merely characterizing known data and have enabled
us to predict \textit{a priori} the behavior of novel setups which were then
experimentally tested and shown to conform to those predictions.

This work aims to extend the above analysis to include another central aspect of
simple repression, namely, the feedback mechanism by which the \textit{lac}
system responds to changing levels of lactose \cite{JACOB1961}. Even in the
absence of glucose, the Lac repressor will continue to bind to the \textit{lac}
operon and block transcription unless bacterial cells are in the presence of
lactose. When lactose is available, it occasionally gets transglycosylated by
$\beta$-galactosidase into allolactose, an inducer which binds to the Lac
repressor and hinders its ability to bind to DNA, thereby allowing the cell to
transcribe the lactose-digesting machinery present in the \textit{lac} operon.
In this work, we will study induction of the Lac repressor using isopropyl
$\beta$-D-1-thiogalactopyranoside (IPTG), a non-hydrolyzable analog of
allolactose.

Specifically, we extend the previous thermodynamic models for simple repression
to include the effects of an inducer. This introduces three new parameters into
our model, namely the binding affinity of inducer to the Lac repressor in the
two allosteric conformations as well as the difference in free energy between
the repressor's allosteric states, and we discuss how these parameters can be
fit to a single inducer titration curve. With these new parameters in hand, we
can quantitatively predict the induction profile while tuning the other
experimental knobs without no additional fit parameters. We create more than 20
\textit{E. coli} mutants with different repressor copy numbers and promoter
binding affinity and show that in each case, the theoretical predictions match
the experimental measurements. Because these mutants are all governed by the
same model of transcription, we can collapse all of the data onto a single
master curve. This data collapse represents a statistical mechanical model
provides a single, unifying framework with which to understand simple
repression.

These results demonstrate that thermodynamic models provide an important vantage
to quantify the workings of the Lac repressor. Furthermore, we have shown that
adding the additional feature of induction is consistent with previous results
on other tunable parameters such as gene copy number and promoter binding
affinity affect gene expression. This gives confidence that with a solid
understanding of the fundamental building blocks, we can accurately model
complex, multi-step systems.


%%%%%%%%%%%%%%%%%%%%%%%%%%%%%%%%%%%%%%%%%%%%%%%%%%%%%%%%%%%%%%%%%%%%%%%%%%%
\section*{Results}

\subsection*{The Monod-Wyman-Changeux (MWC) Model} 

\talComment{Give some background on previous experiments on Lac repressor, and definitely mention the ones from our group: \cite{Garcia2011, Brewster2014, Weinert2014}. Must also quote Mitch Lewis: \cite{Daber2011a, Daber2009}}

\textit{Mention at some point:} The repressor fluctuates between two
allosteric conformations, an active state which binds strongly to the Lac
operator and an inactive state with weak DNA affinity.

\talComment{Continue to modify text below!} The ability of the the Lac repressor
to regulate transcription has been previously characterized by an equilibrium
model where the probability of each state of repressor and RNA polymerase
occupancy is proportional to its Boltzmann weight \cite{Daber2011a,
	Phillips2015a}. We begin with a summary of this model. Suppose there are \(P\)
RNA polymerase (RNAP) and \(R\) repressor molecules in a cell. \(R_A\)
repressors will be in the active state (the favored state when repressor is not
bound to inducer; in this state the repressor binds tightly to DNA) and the
remaining \(R_I\) repressors will be in the inactive state (the predominant
state when repressor is bound to inducer; in this state the repressor binds
weakly to DNA) so that \(R_A+R_I=R\).

We first model the interaction between the Lac repressor and DNA by enumerating
all possible states and their corresponding weights. As shown in
\fref[figpolymeraseRepressorStates], the Lac promoter can either be empty,
occupied by RNAP, or occupied by either an active or inactive repressor
molecule. Assume that there are $N_{NS}$ non-specific sites on the DNA outside
the Lac operator where RNAP or the Lac repressor can bind. \(\Delta\epsilon_{PD}\)
represents the energy difference between RNAP bound to the Lac operator or bound
elsewhere on the DNA; \(\Delta\epsilon _{RD,A}\) and \(\Delta\epsilon _{RD,I}\) equal the
difference in energy when the Lac repressor is bound to the Lac operator
compared to when it is bound non-specifically elsewhere on the DNA in the active
and inactive state, respectively. $\beta = \frac{1}{k_BT}$ where $k_B$ is
Boltzmann's constant and $T$ is the temperature of the system.

\begin{figure}[h]
	\centering \includegraphics[scale=\globalScalePlots]{figure1v2.pdf}
	\caption{{\bf States and weights for simple repression.} 
		\talComment{Placeholder figure and caption. Need a good representation of the Lac repressor with two binding sites on each dimer. We should also probably make both of the dimers inactive in the bottom state to prevent confusion (in the next figure we can show all states as needed)}
		States and weights for an empty, RNAP-bound, and
		active/inactive-repressor-bound Lac promoter. Note the factor of 2 in the
		repressor weights. Not all possible repressor states are shown, but they are
		all accounted for by the numbers $R_A$ and $R_I$ of active and inactive
		repressors, respectively.
%		Both RNAP (light blue)
%		and repressor compete for DNA binding. There are $R_A$ repressors in the active
%		state (green, sharp) and $R_I$ repressors in the inactive state (green,
%		rounded), with the latter type typically bound to inducer (gold). The
%		difference in energy between a repressor bound to the Lac operator and to
%		another non-specific site on the DNA equals $\Delta\epsilon_{RD,A}$ in the
%		active state and $\Delta\epsilon_{RD,I}$ in the inactive state; the $P$ RNAP
%		have a corresponding energy difference $\Delta\epsilon_{PD}$. The number of
%		active repressors $R_A$ includes repressors that are unbound, singly bound, or
%		doubly bound to inducer, although the majority of active state repressors will
%		not be bound to inducer (which pushes them into the inactive state). Similarly,
%		the $R_I$ term includes all inactive state repressors bound to any number of
%		inducer molecules, with the most prevalent state shown in the figure.
		}
	\label{figpolymeraseRepressorStates}
\end{figure}

\talComment{I am personally advocating shoving this derivation to an Appendix. Introducing terms such as $p$, $r_A$, $r_I$ which the reader will never see again is just wasting brain space in their attempt to understand the paper. We should just quote the definition of fold-change and then skip to the result, with the details in an Appendix.}

In thermodynamic models of transcription, gene expression is proportional to the
probability $p_{\text{bound}}$ that RNAP is bound to the Lac operator which is
given by
\begin{equation}\label{eq2}
p_{\text{bound}}(R)=\frac{p}{1+2r_A+2r_I+p},
\end{equation}
where
\begin{align}
p &= \frac{P}{N_{\text{NS}}}e^{-\beta  \Delta\epsilon _{PD}} \\
r_A &= \frac{R_A}{N_{\text{NS}}}e^{-\beta \Delta\epsilon _{RD,A}} \\
r_I &= \frac{R_I}{N_{\text{NS}}}e^{-\beta  \Delta\epsilon _{RD,I}}.
\end{align}
Note the factor of 2 in the repressor states, which arises from the possibility
that either repressor dimer may bind to DNA. Gene expression can be readily
measured experimentally by exploiting the fold-change,
\begin{equation}\label{eq3}
\foldchange\equiv \frac{p_{\text{bound}}(R)}{p_{\text{bound}}(0)}=\frac{1+p}{1+2r_A+2r_I+p}.
\end{equation}
We can simplify this expression using two well-justified approximations: \(p\ll
1\) and \(r_I\ll r_A\). The first approximation is called the weak promoter
approximation and is valid for the wild type Lac promoter \cite{Brewster2012}.
The second approximation follows because $e^{-\beta  \Delta\epsilon _{RD,I}}$ is
approximately 1000 times smaller than $e^{-\beta \Delta\epsilon _{RD,A}}$ for
the Lac repressor \cite{Daber2011a}. Using these approximations, the fold-change
reduces to the form
\begin{equation}\label{eq4}
\foldchange\approx \frac{1}{1+2r_A}=\left(1+\frac{2R_A}{N_{\text{NS}}}e^{-\beta  \Delta\epsilon _{RD,A}}\right)^{-1}.
\end{equation}

We now introduce the role of inducer binding, which changes the number of
repressors in the active and inactive allosteric states. We define \(p_A(c)
\equiv \frac{R_A(c)}{R}\) to be the fraction of repressors in the active state
given a concentration \(c\) of the inducer IPTG. We define \(V\) as the volume
of an \textit{E. coli} cell, \([R]=\frac{R}{V}\) as the concentration of
repressors, and \(\K=\frac{N_{\text{NS}}}{V}e^{\beta \Delta\epsilon _{RD,A}}\)
as the dissociation constant of the active repressor binding to the Lac operator
DNA. This last expression, which links the physical energies of the system with
the language of dissociation constants and chemical rates, is discussed in
detail in the Supplementary Information. With these definitions, \eref[eq4]
becomes
\begin{align}
\foldchange &= \left( 1+\frac{2p_A(c) [R] V}{N_{\text{NS}}}e^{-\beta  \Delta\epsilon
	_{RD,A}} \right)^{-1} \nonumber \\
&= \left( 1+\frac{2p_A(c) [R]}{\K} \right)^{-1}. \label{eq5}
\end{align}

The Lac repressor is a tetrameric protein, a dimer of dimers, with four
identical binding sites for an inducer. It is unknown whether the two dimers are independently allosteric (i.e. whether one dimer can be active while the other is inactive) or allosterically linked. In this paper we adopt the former viewpoint.

As shown in \fref[figrepressorInducerStates], we can enumerate the relative
likelihood of the eight possible conformations of the repressor (the repressor
can be in an active or inactive state, and each of its two inducer binding sites
can be empty or occupied), using the energy difference $\epsilon$ between a Lac
repressor in the active and inactive state. From these eight states, we can
compute the probability \(p_A(c)\) that a repressor will be in the active state
as the sum of the weights of the active states divided by the sum of the weights
of every possible state, namely,
\begin{equation}\label{eq6}
p_A(c)=\frac{\left(1+\frac{c}{K_A}\right)^2}{\left(1+\frac{c}{K_A}\right)^2+e^{-\beta  \epsilon }\left(1+\frac{c}{K_I}\right)^2}.
\end{equation}

\begin{figure}[h]
	\centering \includegraphics[scale=\globalScalePlots]{figure2.pdf}
	\caption{{\bf The eight states of the Lac repressor.} \talComment{Placeholder graphics, needs to be updated to tetrameric form. I envision having the top dimer slightly opaque since we don't want to needlessly write out all of its states as well. We just have the 8 states for the bottom dimer and then state that the states of the top dimer factor out when computing pActive.} The Lac repressor (green)
		has an active conformation (left column) and inactive conformation (right
		column), with the energy difference between these two states given by
		$\epsilon$. In each conformation, the repressor can bind an inducer (gold) at
		two sites. Each state is shown with its corresponding Boltzmann weight. If the
		sum of the active state weights shown (bottom left) is greater than the sum of
		the inactive state weights (bottom right), the repressor is more likely to be
		in the active state.} \label{figrepressorInducerStates}
\end{figure}

Substituting this result into \eref[eq5] yields the complete formula
\begin{equation}\label{eq7}
\foldchange= \left(
1+\frac{\left(1+\frac{c}{K_A}\right)^2}{\left(1+\frac{c}{K_A}\right)^2+e^{-\beta
		\epsilon }\left(1+\frac{c}{K_I}\right)^2}\frac{2[R]}{\K} \right)^{-1},
\end{equation}
which predicts that given a concentration \([R]\) of Lac repressor and a
concentration \(c\) of the inducer IPTG, the fold-change in gene expression will
depend solely on 4 parameters: the DNA binding affinity of the repressor
(\(\K\)), the inducer binding affinities for the repressor in the active state
(\(K_A\)) and inactive state (\(K_I\)), and the difference in free energy
between the active and inactive states of the repressor (\(\epsilon\)).

%%%%%%%%%%%%%%%%%%%%%%%%%%%%%%%%%%%%%%%%%%%%%%%%%%%%%%%%%%%%%%%%%%%%%%%%%%%
\section*{\textcolor{blue}{To Do}}

\begin{enumerate}
	\item Discussion of $\epsilon$ parameter and the implications of its sloppiness. The other parameters are fixed once you nail down $\epsilon$. Introduce Rob Brewster's data  which shows (using Stephanie's version 1 or version 2) that it is probably at least $\epsilon > 0$, but this can not be determined exclusively. However, once you choose a parameter value, the other parameters will adjust to accommodate this.
	\begin{enumerate}
		\item Question: Will the mutants data be able to nail this down further?
	\end{enumerate}
	
	\item We should also test the model of what happens if the two repressor dimers are allosterically linked, so that one cannot be active while the other is inactive. Would this still be consistent with the data? If so, maybe need to add an Appendix section with these types of fits. If not, maybe need to add an Appendix section detailing why the data negates this hypothesis.
	
	\item \talComment{I think the derivations should be relegated to the Appendix, but probably not initially}
	
	\item How do we deal with the $\epsilon$ parameter? State that to ultimately nail this down, NMR has become the standard tool to directly measure the $\epsilon$ parameter \cite{Gardino2003, Boulton2016}.
	\begin{itemize}
		\item Stephanie's trick version 1 yields $\beta \epsilon = 4.5$, and her second trick yields something positive (but probably impossible to nail down from the current data). This explanation should probably be relegated to Appendix \talComment{but it would be great to be able to also use the $x$-measurement error in the fit to Rob's data!}
	\end{itemize}
	
	\item Get proper error on fit parameters. MCMC to get credible region (it would be interesting if this is significantly different from the confidence interval)?
\end{enumerate}

%%%%%%%%%%%%%%%%%%%%%%%%%%%%%%%%%%%%%%%%%%%%%%%%%%%%%%%%%%%%%%%%%%%%%%%%%%%
%%%%%%%%%%%%%%%%%%%%%%%%%%%%%   APPENDIX   %%%%%%%%%%%%%%%%%%%%%%%%%%%%%%%%
%%%%%%%%%%%%%%%%%%%%%%%%%%%%%%%%%%%%%%%%%%%%%%%%%%%%%%%%%%%%%%%%%%%%%%%%%%%
\appendix

%%%%%%%%%%%%%%%%%%%%%%%%%%%%%%%%%%%%%%%%%%%%%%%%%%%%%%%%%%%%%%%%%%%%%%%%%%%
\section{Sloppiness} \label{AppendixSloppiness}

\talComment{An initial discussion of sloppiness and the $\epsilon$ parameter. Needs to be updated with one of our actual data sets. Perhaps if we have enough data sets this sloppiness will go away?}

We have all seen by now that we can wiggle the best fit parameters around and still get good fitting. To better visualize the problem, I took Stephanie's 2016-06-18 data for HG 104 and RBS 1027 and fit it to the form \eref[AppendixSloppinessEq1], forcing the $\beta \epsilon_R$ to vary between -10 and 10. \fref[SIfig1] shows the results.
\begin{figure}[h]
	\centering \includegraphics{SIfigure1.pdf}
	\caption{{\bf Many best fit parameters generate good fits.} \letterParen{A} I forced $\beta \epsilon_R$ to take on a particular value and fit $K_A$ and $K_I$ to the data using \eref[AppendixSloppinessEq1]. All of the resulting fit curves look good. \letterParen{B} The best fit $K_A$ and $K_I$ values. This graph seemed quite provacative to me, since the values of $K_A$ and $K_I$ seem to be linear curves on this logarithmic plot.}
	\label{SIfig1}
\end{figure}

Rather than continuing to beat around the bush, I thought it was time to tackle this issue head on. The following analysis is for the following $n=2$
exponent form,
\begin{align} \label{AppendixSloppinessEq1}
\foldchange &= \frac{1}{1+\frac{\left( 1 + \frac{c}{K_A} \right)^2 \left(1 + e^{-\beta \epsilon_R} \right)}{\left( 1 + \frac{c}{K_A} \right)^2 + e^{-\beta \epsilon_R} \left( 1 + \frac{c}{K_I} \right)^2} \frac{2 \Rtot}{N_{NS}} e^{-\beta \Delta\epsilon_{\text{O2}}^{(\text{Hernan})}}}\nonumber\\
&\equiv \frac{1}{1+\frac{\left( 1 + \frac{c}{K_A} \right)^2 \left(1 + e^{-\beta \epsilon_R} \right)}{\left( 1 + \frac{c}{K_A} \right)^2 + e^{-\beta \epsilon_R} \left( 1 + \frac{c}{K_I} \right)^2} r},
\end{align}
where for notational convenience I have defined $r = \frac{2 \Rtot}{N_{NS}}
e^{-\beta \Delta\epsilon_{\text{O2}}^{(\text{Hernan})}}$ (with $r \approx 1$ for
HG 104, using Hernan's values). Recall that $K_I < K_A$, since the inducer makes
the repressor more likely to assume the inactive state. I will analyze this
fold-change equation in the two limits: $e^{-\beta \epsilon_R} \ll 1$ and
$e^{-\beta \epsilon_R} \gg 1$.

\subsection{$\boldsymbol{e^{-\beta \epsilon_R} \ll 1}$}

In the limit $e^{-\beta \epsilon_R} \ll 1$, the fold-change equation becomes
\begin{align} \label{AppendixSloppinessEq2}
\foldchange &= \frac{1}{1+\frac{\left( 1 + \frac{c}{K_A} \right)^2}{\left( 1 + \frac{c}{K_A} \right)^2 + e^{-\beta \epsilon_R} \left(\frac{c}{K_I} \right)^2} r} \nonumber\\
&\equiv \frac{1}{1+\frac{\left( 1 + \frac{c}{K_A} \right)^2}{\left( 1 + \frac{c}{K_A} \right)^2 + \left(\frac{c}{\widetilde{K}_I} \right)^2} r}.
\end{align}
where we have defined $\widetilde{K}_I = K_I e^{\beta \epsilon_R/2}$. This
suggests that what we are actually fitting $K_A$ and $\widetilde{K}_I$ in this
regime, as confirmed by \fref[SIfig2] in the region $e^{-\beta \epsilon_R} \ll 1$ (note that the $K_I e^{\beta \epsilon_R/2}$ values do not change).
%where we have made two approximations. First, we assumed $\left(1 + e^{-\beta
% \epsilon_R} \right) \approx 1$ in the numerator. Second, we used $\left( 1 +
% \frac{c}{K_A} \right)^2 + e^{-\beta \epsilon_R} \left( 1 + \frac{c}{K_I}
% \right)^2 \approx \left( 1 + \frac{c}{K_A} \right)^2 + e^{-\beta \epsilon_R}
% \left(\frac{c}{K_I} \right)^2$ in the denominator because the $e^{-\beta
% \epsilon_R} \left(1 + \frac{c}{K_I} \right)^2$ term
\begin{figure}[h]
	\centering \includegraphics{SIfigure2.pdf} \caption{{\bf The altered
			form of $K_I$ we are actually fitting when $\boldsymbol{e^{-\beta \epsilon_R}
				\ll 1}$.} Redrawing of \fref[SIfig1]\letter{B} showing $K_I$ and $\widetilde{K}_I
		= K_I e^{\beta \epsilon_R/2}$, the latter forming a straight line when
		$e^{-\beta \epsilon_R} \ll 1$.} \label{SIfig2}
\end{figure}

With only two effective parameters, you should be able to figure out their
values without doing any fitting. To do this, consider three important
characteristics of a titration curve: (1) the \textit{leakiness} equals the fold-change
when $c=0$; (2) the \textit{dynamic range} equals the difference between the maximal
fold-change at $c\to\infty$ and $c=0$; and $[EC_{50}]$ equals the concentration at
which fold-change reaches halfway between its minimum and maximum values.

Using \eref[AppendixSloppinessEq1], 
\begin{align}
\lim_{c = 0}\foldchange &= \frac{1}{1+r}\\
\lim_{c \to \infty}\foldchange &= \frac{\left(\frac{K_A}{\widetilde{K}_I}\right)^2 + 1}{\left(\frac{K_A}{\widetilde{K}_I}\right)^2 + 1+r}
\end{align}
so that
\begin{align}
\text{leakiness} &= \frac{1}{1+r} \label{AppendixSloppinessEq3} \\
\text{dynamic range} &= \frac{\left(\frac{K_A}{\widetilde{K}_I}\right)^2 + 1}{\left(\frac{K_A}{\widetilde{K}_I}\right)^2 + 1+r} - \frac{1}{1+r}. \label{AppendixSloppinessEq4}
\end{align}
Since you can read the leakiness and dynamic range straight off the data, \eref[AppendixSloppinessEq4] already gives us one relationship for $K_A$ and $\widetilde{K}_I$,
\begin{equation} \label{AppendixSloppinessEq5}
\frac{K_A}{\widetilde{K}_I} = \left( \left(1+r\right) \frac{\text{dynamic range} - \frac{1}{1+r}}{1-\text{dynamic range}} \right)^{1/2}.
\end{equation}

We next solve for $[EC_{50}]$, showing first an exact formula and then an approximation that can be achieved by noting that $\widetilde{K}_I < K_A$ and that $r \approx 1$,
\begin{align}
[EC_{50}] &= \frac{K_A \widetilde{K}_I^2 \left(1+r\right) + \sqrt{K_A^2 \widetilde{K}_I^2 \left(1+r\right) \left(K_A^2 + 2 \widetilde{K}_I^2 \left(1+r\right) \right)}}{K_A^2 + \widetilde{K}_I^2 \left(1+r\right)}\nonumber\\
&\approx \widetilde{K}_I \left(1+r\right)^{1/2}.
\end{align}
Combining this formula with \eref[AppendixSloppinessEq5], we can directly find the value of our two parameters without resorting to fitting,
\begin{align}
\widetilde{K}_I &= \frac{[EC_{50}]}{\left(1+r\right)^{1/2}}\\
K_A &= [EC_{50}] \left(\frac{\text{dynamic range} - \frac{1}{1+r}}{1-\text{dynamic range}} \right)^{1/2}.
\end{align}


\subsection{$\boldsymbol{e^{-\beta \epsilon_R} \gg 1}$}

In this limit, \eref[AppendixSloppinessEq1] becomes 
\begin{equation}
\foldchange = \frac{1}{1+\left(\frac{1 + \frac{c}{K_A}}{1 + \frac{c}{K_I}}\right)^2 r},
\end{equation}
since in the numerator $\left( 1 + e^{-\beta \epsilon_R} \right) \approx
e^{-\beta \epsilon_R}$ while in the denominator the
$\left(1+\frac{c}{K_A}\right)^2$ term is always smaller than the $e^{-\beta
	\epsilon_R} \left(1+\frac{c}{K_I}\right)^2$ term. In this case, we are also
fitting only two parameters (only $K_A$ and $K_I$), and we see that the $\beta
\epsilon_R$ value should not influence these parameters, as is confirmed in
\fref[SIfig1]\letter{B}.

You can play the same game as in the previous section to get the values of these two parameters from the dynamic range and $[EC_{50}]$. For now, I skip the details and just write the results,
\begin{align}
K_I &= \frac{[EC_{50}]}{\left(2+r\right)^{1/2}-1}\\
K_A &= [EC_{50}] \frac{r^{1/2}}{\left(2+r\right)^{1/2}-1} \left(\frac{\text{dynamic range}}{1-\text{dynamic range}}\right).
\end{align}

\subsection{Thoughts}

It appears that in either regime, we are really fitting two parameters, and a
third parameter is ``going along for the ride.'' I am not sure how to determine
whether $e^{-\beta \epsilon_R} \ll 1$ or $e^{-\beta \epsilon_R} \gg 1$, since
both regimes yield great (and very similar) fits. However, within the two
regimes I now think that I understand exactly what $K_A$ and $K_I$ parameters
will come out of fitting. We need another insight to get us the sign of $\beta
\epsilon$, and then I think the sloppiness in our best-fit parameters will be
much more under control.

\nolinenumbers

%%%%%%%%%%%%%%%%%%%%%%% BIBLIOGRAPHY %%%%%%%%%%%%%%%%%%%%%%%%%%%%
% Compile your BiBTeX database using our plos2009.bst
% style file and paste the contents of your .bbl file
% here.

\bibliography{library}


\end{document}

