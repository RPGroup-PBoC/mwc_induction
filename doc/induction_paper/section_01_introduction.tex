%%%%%%%%%%%%%%%%%%%%%%%%%%%%%%%%%%%%%%%%%%%%%%%%%%%%%%%%%%%%%%%%%%%%%%%%%%%
\section*{Introduction \stephanieComment{Stephanie's Section!}}

Biological mechanisms are complex and diverse, and are often characterized by a
slow accumulation of qualitative data \nathanComment{maybe include 'their
understanding' is often characterized by ... }. Our understanding of biological
systems can be expanded greatly by applying quantitative, predictive models to
pervasive phenomena that, in the absence of models, exist as a collection of
loosely-associated facts. Allosteric conformational change is an example of
such a pervasive phenomenon. Allosteric proteins switch between conformational
states in response to ligand binding. Perhaps most famously characterized in
hemoglobin, allostery is also very common among transcription factors that
switch between "active" and "inactive" states with significantly different
regulatory capabilities. Here, we use quantitative models to describe
transcriptional regulation through allosteric induction, and show that such
models can be used to predict the behavior of novel systems.\nathanComment{
With the first few lines, I worry that it'll come off as too belittling to
non-quantitative approaches}\griffinComment{I usually dislike the idea that anything not quantitative is useless. I agree that we should soften it. We should emphasize that if we can predict the behavior of the system from a very general point of view, we understand.}

As the theoretical basis for our models, we turn to what may be the most powerful model of allostery, the Monod-Wyman-Changeux (MWC) model \cite{MONOD1965}. \griffinComment{We need to standardize the cite keys} Briefly, the MWC model stipulates that a symmetrical allosteric protein is capable of accessing at least two states, a "tense" and "relaxed" state (or, as we will refer to them here, an "active" and "inactive" state). These states exist in equilibrium within a system, and a protein may switch between states at any time. Both states are capable of binding to a ligand, though the protein's affinity for the ligand will be affected significantly by the protein's allosteric conformation. Given a high enough concentration of ligand, it becomes energetically favorable for a protein to spend the significant majority of its time in the state that favors ligand binding, thus shifting the overall protein population towards the ligand-favoring state. We have previously discussed the statistical mechanics of MWC models \cite{Marzen2013}, and we use such a statistical mechanical framework for our model of transcriptional induction. \nathanComment{we say statistical mechanics a lot. Maybe remove the second one in this sentence.}

To provide a straightforward context for exploring allosteric induction \textit{in vivo}, we use a chromosomally-integrated simple repression construct in which a single repressor binding site blocks access to an RNAP polymerase binding site. A YFP reporter gene downstream of this construct allows us to measure the fold-change caused by repressor binding. As our model repressor, we use the Lac repressor (LacI), which has long been recognized as a rich model  for understanding the nature of allostery within the MWC framework \cite{Daly1986, Dunaway1980, Meyer2013, Daber2007, Daber2009, Muller-Hartmann1996, OGorman1980, Sharp2011, Taraban2008, Wilson2007}. Additionally, in previous studies we have carefully quantified LacI repression as a function of repressor copy number and promoter
affinity \cite{Garcia2011}, as well as gene copy number and competitor binding
sites \cite{Weinert2014}. All of these tunable parameters have been well
characterized by statistical mechanical models \cite{Bintu2005a, Bintu2005b}, allowing us to go beyond merely characterizing known data
and enable us to predict \textit{a priori} the behavior of novel transcriptional
architectures that can then be tested against experimental data.
\griffinComment{Maybe we should leave this section for the results/methods. We could just emphasize that we are testing this theory in an experimental system where we have quantitative control of a number of biophysical parameters.}

This work aims to extend our statistical mechanical models to include
allosteric induction by examining the
feedback mechanism by which the \textit{lac} system responds to changing levels
of lactose \cite{JACOB1961}. We extend the previous thermodynamic models for
simple repression to include the effects of an inducer. This introduces three
new parameters into our model: the binding affinity of inducer to LacI in the
``active'' or DNA-binding conformation, the binding affinity of inducer to LacI
in the ``inactive'' or non-DNA-binding conformation, and the free energy
difference between these two conformations. We discuss how these parameters can
be inferred by fitting our model to repression data acquired under varying
concentrations of IPTG \nathanComment{Since we tend to talk about fold-change,
or fold-change in expression, maybe used 'expression' instead of 'repression'
data}. With these new parameters in hand, we quantitatively
predict the induction profile for other strains in which operator affinity and
LacI copy number have been varied. We analyze more than 20 such mutants and show
that in each case, the theoretical predictions match the experimental
measurements. In addition, we show that the mechanisms governing fold-change in
each mutant are systematically equivalent by identifying the ``natural
parameter'' governing IPTG induction; when plotted against this natural
parameter, the fold-change values for all of the mutants fall on the same curve.
In this way, we demonstrate the power of quantitative modeling as a tool for
achieving deep understanding of biological systems. \nathanComment{Switches from
strain to mutant half way through. I'd avoid using mutant}
