%%%%%%%%%%%%%%%%%%%%%%%%%%%%%%%%%%%%%%%%%%%%%%%%%%%%%%%%%%%%%%%%%%%%%%%%%%%
\section*{Introduction \stephanieComment{Stephanie's Section!}}

The \textit{lac} operon, which is responsible for the metabolism of lactose in
\textit{Escherichia coli}, is arguably one of the best studied sets of genes
genes of all time \cite{MullerHill1996a}. Due to its central importance in areas
of biology ranging from transcription to evolution to allostery
\cite{Oehler1994, Razo-Mejia2014, Daber2008, Lakshmi2009, Swint-Kruse2009,
	Garcia2010, Boedicker2013, Brewster2014, Kuhlman2007}, a vast amount of data has
been collected for this system which has given rise to many quantitative models.

The Lac repressor (LacI) is the key molecule which binds to the \textit{lac}
promoter and inhibits gene expression in the absence of effector. When glucose
is absent and lactose is available, lactose is converted to allolactose by
$\beta$-galactosidase. When allolactose binds to the effector-binding pocket of
LacI, LacI assumes a conformation with significantly reduced DNA-binding
affinity. This allows the \textit{lac} operon to be transcribed, enabling
lactose metabolism.

Shortly after LacI's isolation 50 years ago \cite{Gilbert1966}, efforts began to
thoroughly characterize its interactions with DNA operators and small-molecule
effectors \cite{Riggs1970I, Riggs1970II, Riggs1970III, Jobe1972}. During this
time it was discovered that although allolactose is the natural effector of
LacI, other small molecules could behave as effectors as well, the most notable
among them being isopropyl $\beta$-D-1-thiogalactopyranoside (IPTG). As an
allosteric protein with two distinct binding states, it was soon realized that
LacI could be described well using the Monod-Wyman-Changeux (MWC) model of
allostery \cite{MONOD1965}. LacI has proven to be a rich model for understanding
the nature of allostery within the MWC framework \cite{Daly1986, Dunaway1980,
	Meyer2013, Daber2007, Daber2009, Muller-Hartmann1996, OGorman1980, Sharp2011,
	Taraban2008, Wilson2007}.

In recent years, the Lac repressor's ability to inhibit gene expression has been
carefully quantified as a function of repressor copy number and promoter
affinity \cite{Garcia2011}, as well as gene copy number and competitor binding
sites \cite{Weinert2014}. All of these tunable parameters have been well
characterized by statistical mechanical models \cite{Bintu2005a, Bintu2005b}.
Importantly, such models allow us to go beyond merely characterizing known data
and enable us to predict \textit{a priori} the behavior of novel transcriptional
architectures that can then be tested against experimental data.

Although statistical mechanical models of the \textit{lac} operon have deepened
our understanding of many of the parameters controlling transcriptional
regulation, these models have remained largely agnostic of allostery in the Lac
repressor. This work aims to extend our statistical mechanical models to include
this central aspect of allosteric transcription factors by examining the
feedback mechanism by which the \textit{lac} system responds to changing levels
of lactose \cite{JACOB1961}. We extend the previous thermodynamic models for
simple repression to include the effects of an inducer. This introduces three
new parameters into our model: the binding affinity of inducer to LacI in the
``active'' or DNA-binding conformation, the binding affinity of inducer to LacI
in the ``inactive'' or non-DNA-binding conformation, and the free energy
difference between these two conformations. We discuss how these parameters can
be inferred by fitting our model to repression data acquired under varying
concentrations of IPTG. With these new parameters in hand, we quantitatively
predict the induction profile for other strains in which operator affinity and
LacI copy number have been varied. We analyze more than 20 such mutants and show
that in each case, the theoretical predictions match the experimental
measurements. In addition, we show that the mechanisms governing fold-change in
each mutant are systematically equivalent by identifying the ``natural
parameter'' governing IPTG induction; when plotted against this natural
parameter, the fold-change values for all of the mutants fall on the same curve.
In this way, we demonstrate the power of quantitative modeling as a tool for
achieving deep understanding of biological systems.

