%%%%%%%%%%%%%%%%%%%%%%%%%%%%%%%%%%%%%%%%%%%%%%%%%%%%%%%%%%%%%%%%%%%%%%%%%%%
\section*{Introduction}

\talComment{Dear Team Induction. After re-reading Rob's comments, I think that
	we should not mention Lac at all in this introduction. Let's keep it totally
	general, and then in the Results section we can bring up the specific genetic
	circuit that we used to test our theory.}

---

%paragraph 1: the ubiquitous induction phenomenon -  a key part of the
%central dogma story that is the meeting point of cell signaling and
%gene regulation.  I would also touch on the “Whence Cometh the Allosterome”
%paper from PNAS that notes how hard it is to do any sort of high
%throughput analysis of allostery.

The communication of organisms with the external world is one of the most
ubiquitous phenomena in biology. At the cellular level, this takes the form of
\talComment{diverse} signaling pathways \talComment{from two component systems
	found primarily in bacteria to multi-component pathways in eukaryotes involving
	MAPK, cAMP, IP$_3$, and other key signaling molecules}. One of the most
pervasive categories \talComment{found in all realms} of cell signaling is
allosteric regulation, where binding of a ligand induces a conformational change
in transcription factors, dictate gene expression. Despite the overarching
importance of this mode of signaling, we have lacked a rigorous dialogue between
theory and experiment \cite{Lindsley2006}. To that end, we explore cell
signaling in the context of one of the most widespread bacterial regulatory
architectures, the simple repression motif \cite{Rydenfelt2014}.

%Paragraph 2: The challenge we undertake - a predictive theory of the response of
%genetic circuits to allosteric inputs.  What we require of such a predictive
%theory is not that one can fit some sigmoidal curve, but rather something that
%allows us to understand all aspects ranging from leakiness to the EC50 and
%effective Hill coefficient to the dynamic range.

---

\textbf{Paragraph 2}

Here we propose a predictive theory of the response of genetic circuits to
allosteric inputs. This theory allows us to understand and predict the fine
details of an induction profile -- such as the steepness of the response and
the limits of minimum and maximum expression -- beyond a general sigmoidal
shape.

\talComment{In this work, we demonstrate that a thermodynamic model of transcriptional
regulation coupled with a statistical mechanical treatment of the induction of
allosteric proteins allows us to go beyond merely understanding the sigmoidal
shape of induction profiles, but allows us to quantitatively predict fine
details of such responses, such as the steepness of the response as well as the
limits of minimum and maximum expression.}

% The inputs to our model must therefore have a strong biophysical basis.
% A strong theory of allosteric induction is likewise required to place each of
% these inputs in their proper context.
%
%
% \griffinComment{How about this? "The
% input into our model must therefore have meaning that obeys biophysical
% reasoning. A strong theory of allosteric induction is imperative to place these
% inputs in their proper context."}

---

\textbf{Paragraph 3}

%Paragraph 3: We show a general statistical mechanical theory of the induction
%phenomenon using the famous MWC model. We show how this model can be used for
%various transcription factors and modes of induction (i.e. how many inducer
%binding sites, whether inducer switches from active to inactive or the other way
%around).

\sout{As the foundation for our theory of allosteric induction, we use a
	statistical mechanical rendering of} \talComment{awk} the Monod-Wyman-Changeux
(MWC) model, which stipulates that an allosteric protein is capable of accessing
two distinct conformations -- an active and inactive state --
\griffinComment{fluctuating between them} in thermodynamic equilibrium
\cite{MONOD1965}. \griffinComment{For the existence of these states to be
	useful, a} protein's affinity for ligand is different for the active and
inactive states, such that the equilibrium between these states shifts with
changing concentration of ligand \cite{Marzen2013}. We show how this model can
be used for various transcription factors and modes of induction, accounting for
variations such as the number of ligand binding sites, the regulatory effect of
the ligand (i.e. whether it switches the transcription factor from active to
inactive or vice versa), and cooperativity \talComment{When do we discuss
	cooperativity???} \griffinComment{Aren't we assuming infinite cooperativity
	here? We assign a arbitrarily large energetic penalty for dependent subunits to
	be in different active/inactive states} between transcription factor subunits.
\stephanieComment{At this point we don't actually discuss variations to the
	allosteric properties of the transcription factor, but I like the idea of
	addressing it by showing theory curves for hypothetical TFs with various
	properties. This could either be part of our primary results or a cool appendix
	for the SI.}\griffinComment{I think Rob would even like this to be in the model
	section of the text. He had mentioned talking about the case of activation and
	induction of repression. He also expressed interest in having some kind of small
	figure which shows the numerous ways our theory can be applied. Maybe that is
	better saved for a review article, however.}

\talComment{Specifically, we consider allosteric transcription factors using the
Monod-Wyman-Changeux (MWC) model of allostery, which stipulates that an
allosteric protein is capable of accessing two distinct conformations -- an
active and inactive state -- in thermodynamic equilibrium \cite{MONOD1965}. For
example, an inducer binding to an allosteric repressor increases the probability
that the repressor will be inactive, weakening its ability to bind to the
promoter and increasing gene expression. We discuss how such frameworks are
remarkably general, capable of modeling both induction and corepression in
addition to accounting for the number of binding sites and binding energies of a
system.}


---

\textbf{Paragraph 4}

%Paragraph 4: we design and implement an experimental system in which
%we tune many distinct parameters in the genetic circuit itself - i.e. the binding
%energies and transcription factor copy numbers.  We pose a simple question.
%Once we have determined the three parameters that characterize
%the transcription factor, can we then use that completely characterized
%transcription factor in all of these different contexts and predict what the outcome
%will be.

To show how our model accounts for the biophysical parameters unique to each
individual regulatory motif, we use the well-characterized components of the
\textit{lac} regulatory system. We design and implement an experimental system
in which we tune several distinct parameters in the genetic circuit itself,
such as the transcription factor binding energies and copy numbers. This leaves
the allosteric properties of the transcription factor as the only free
parameters, which should remain constant in each of our variations on the
genetic circuit \talComment{awk}. We then assess the predictive nature of our
model with the following question: once we have determined the \sout{three} parameters
\talComment{We should not say three parameters unless we explicitly state what
they are} that characterize the allosteric properties of the transcription
factor, can we then apply these parameters to each of our designed genetic
circuits and \sout{determine}\griffinComment{predict} the regulatory outcome?

\talComment{To test our model, we design a genetic circuit which allows us to
	directly tune various physical parameters associated with a transcription factor
	including its copy number and DNA binding energy. We then pose a simple
	question: after we characterize the biophysical parameters associated with
	induction, how much predictive power does our model provide when we tune the
	parameters associated with the transcription factor? To this end, we construct
	24 genetic circuits, each with a different combination of transcription factor
	copy number and DNA binding energy. We use data from a single strain in order to
	completely characterize the system and make parameter-free predictions of the
	remaining 23 strains.}

---

\textbf{Paragraph 5}

%Paragraph 5: We show the measurements, the comparison between the theory
%and the experiment and propose a new fundamental parameter that allows
%us to see how there are many ways to reach the same level of transcription
%and all of these are equivalent.

To \sout{answer this question}\griffinComment{approach this challenge}, we
measure the fold-change values of each of our genetic circuits over several
orders of magnitude of ligand concentration. We show that we can use the ligand
response curve of a single genetic circuit to determine the allosteric
parameters of the transcription factor, and then use these parameters to
accurately predict the response curves for \sout{each of the remaining genetic
circuits}\griffinComment{numerous derivatives of the same regulatory
architecture}. Moreover, we propose a new fundamental parameter (which we call
the Bohr parameter \griffinComment{This is a testament to my ignorance, but are
\textbf{we} actually naming this as the Bohr parameter or has this been done
before?}) that captures the regulatory range of each genetic circuit on a
single single curve, showing that a broad selection of thermodynamic parameters
produce the same regulatory result. \sout{meaning that they are all
biologically equivalent}. \griffinComment{demonstrating that a combination of a
variety of parameter values can yield biologically equivalent responses.}

\talComment{We compare our theoretical predictions with experimental
	measurements and demonstrate how interesting trends such as the minimal and
	maximal responses of the strains can be understood using the MWC framework. In
	addition, we propose a new fundamental parameter which characterizes the
	trade-offs between the different physical parameters of the system. For example,
	if the transcription factor copy number is doubled but its binding energy is
	decreased by $k_B T \log 2$, the level of gene expression would
	remain unchanged. Such relationships not only give insight into the evolution of
	such systems, but also permit us to understand the data from a new perspective.
	We demonstrate how the titration curves from the 24 strains collapse onto a
	master curve. In this manner, the myriad of data from these 24 genetic circuits
	can all be understood as multiple embodiments of the same underlying phenomenon.
	One might imagine extending these models to predict the consequences of other
	perturbations such as mutations, allosteric activation, or more complex
	regulatory architectures.}



%%%%%%%%%%%%%%%%%%%%%%%%%%%%%%%%%%%%%%%
%%% Old introduction, for reference %%%
%%%%%%%%%%%%%%%%%%%%%%%%%%%%%%%%%%%%%%%

%Biological mechanisms are complex and diverse, and as a consequence our
%understanding is often characterized by a slow accumulation of qualitative
%information that is often difficult to fit into a unifying framework.
%Nevertheless, the last decade has yielded new techniques that have given us
%unprecedented access to the fine details underlaying a broad array of biological
%phenomena, from the exquisite dissection of the gene regulatory network of the
%sea-urchin developmental program [Peters and Davidson] to the \textit{in vivo}
%imaging of the central dogma [Ido's MS2, Hernan's MS2, Suntag paper, Xiaowei
%recent paper] to the complex task of regulating gene expression
%\cite{Garcia2011}. New genetic reporters and advancements in single-cell
%measurements have brought about a new level of precision measurements
%traditionally only achievable in more quantitative fields such as physics and
%chemistry. Such high quality data is now routinely generated in various fields
%of biology, and it calls for a similar expansion in our use of theoretical
%frameworks to conceptualize and understand the richness of modern biological
%data sets. At this level, fine details of a system - the steepness of the
%response, the limits of minimum and maximum response, etc. - can now be resolved
%to go beyond a general sigmoidal shapes.
%
%Statistical mechanical models of gene regulation have proven to be very
%effective not only in predicting the mean expression level as a function of
%natural parameters such as the transcription factor copy number and its binding
%energy to the DNA \cite{Garcia2011}, but also in quantifying the effects of DNA
%looping \cite{Boedicker2013a} and competing binding sites \cite{Brewster2014}.
%Here, we extend such models to examine allostery as a mechanism for
%transcriptional regulation. Allostery is a pervasive property of biological
%molecules. This principle helps explain a large array of phenomena ranging from
%quorum sensing and chemotaxis, oxygen binding by hemoglobin, and signal
%transduction via GPCRs. Here, we consider the Monod-Wyman-Changeux (MWC) model
%\cite{MONOD1965}, which stipulates that an allosteric protein is capable of
%accessing two distinct conformations -- an active and inactive state -- in
%thermodynamic equilibrium. A protein's affinity for ligand is different for the
%active and inactive states, such that the equilibrium between these states
%shifts with changing concentration of ligand \cite{Marzen2013}.
%
%In this work, we focus on the rich interplay between theory and experiment in
%transcriptional regulation, where quantitative dissection of input-output
%functions over the last decade has significantly pushed our ability to
%manipulate and understand such systems. We focus on one of the most abundant
%genetic architectures -- the simple repression motif [Mattias paper on
%regulonDB] -- to test our predictions. Specifically, we use a simple repression
%promoter constructed of the well-characterized components of the \textit{E. coli
%	lac} operon and the famously allosteric Lac repressor to quantiatively test a
%statistical mechanical model of induction using the MWC framework.
%
%One remarkable conclusion of this theory is that we can use the response of a
%single strain of known repressor copy number and operator binding energy to make
%parameter free predictions for the induction profile of any other simple
%repression architecture. To test this, we used 20 additional strains in which
%repressor copy numbers and the Lac repressor binding energies have been varied,
%and measured the induction profile of each. The remarkable agreement between
%these 20 predicted sets and the experimental data demonstrates how all of these
%induction profiles collapse onto a master curve governed by a set of underlying
%thermodynamic parameters. These results show that quantitative models are an
%incredibly powerful tool for understanding the myriad ways in which cells
%establish control over their expression profiles. While many combinations of
%parameters are capable of achieving the same fold-change values, these titration
%curves are significantly affected by both repressor copy number and binding
%energy, and we predict these effects neatly with our models. One might imagine
%extending these models to predict the consequences of other perturbations such
%as mutations, allosteric activation, or more complex regulatory architectures.
