%%%%%%%%%%%%%%%%%%%%%%%%%%%%%%%%%%%%%%%%%%%%%%%%%%%%%%%%%%%%%%%%%%%%%%%%%%%
\section*{Introduction}

The lactose (\textit{lac}) system, responsible for the metabolism of lactose in
\textit{Escherichia coli}, is arguably one of the best studied genes of all time
\cite{MullerHill1996a}. Due to its central importance in the areas of biology
ranging from transcription to evolution to allostery \cite{Razo-Mejia2014}
\talComment{More references!}, a vast amount of data has been collected for this
system which has given rise to the many quantitative models.

The Lac repressor, the key molecule which binds to the Lac promoter and inhibits
gene expression, was isolated 50 years ago \cite{Gilbert1966}. It was quickly
understood that the Lac repressor was allosteric, existing in two different
conformations with different DNA binding affinities. Although the Lac repressor
oscillates between these two conformations even in the absence of inducer
\talComment{cite}, the transition between the two states is also mediated by the
binding of an effector to a site roughly 50\AA away from the DNA binding domain
in the folded quaternary structure of the Lac repressor \cite{Lewis1996}. This
two state system was amenable to the Monod-Wyman-Changeux (MWC) model of
allostery \cite{MONOD1965}.

Since then, the Lac repressor's ability to inhibit gene expression has been carefully quantified
as a function of repressor copy number and promoter affinity \cite{Garcia2011}
as well as gene copy number and competitor binding sites \cite{Weinert2014}. All
of these tunable parameters have been well characterized by statistical
mechanical models grounded upon the MWC framework. But importantly, such models
have allowed us to go beyond merely characterizing known data and have enabled
us to predict \textit{a priori} the behavior of novel setups which were then
experimentally tested and shown to conform to those predictions.

This work aims to extend the above analysis to include another central aspect of
simple repression, namely, the feedback mechanism by which the \textit{lac}
system responds to changing levels of lactose \cite{JACOB1961}. Even in the
absence of glucose, the Lac repressor will continue to bind to the \textit{lac}
operon and block transcription unless bacterial cells are in the presence of
lactose. When lactose is available, it occasionally gets transglycosylated by
$\beta$-galactosidase into allolactose, an inducer which binds to the Lac
repressor and hinders its ability to bind to DNA, thereby allowing the cell to
transcribe the lactose-digesting machinery present in the \textit{lac} operon.
In this work, we will study induction of the Lac repressor using isopropyl
$\beta$-D-1-thiogalactopyranoside (IPTG), a non-hydrolyzable analog of
allolactose.

Specifically, we extend the previous thermodynamic models for simple repression
to include the effects of an inducer. This introduces three new parameters into
our model, namely the binding affinity of inducer to the Lac repressor in the
two allosteric conformations as well as the difference in free energy between
the repressor's allosteric states, and we discuss how these parameters can be
fit to a single inducer titration curve. With these new parameters in hand, we
can quantitatively predict the induction profile while tuning the other
experimental knobs without no additional fit parameters. We create more than 20
\textit{E. coli} mutants with different repressor copy numbers and promoter
binding affinity and show that in each case, the theoretical predictions match
the experimental measurements. Because these mutants are all governed by the
same model of transcription, we can collapse all of the data onto a single
master curve. This data collapse represents a statistical mechanical model
provides a single, unifying framework with which to understand simple
repression.

These results demonstrate that thermodynamic models provide an important vantage
to quantify the workings of the Lac repressor. Furthermore, we have shown that
adding the additional feature of induction is consistent with previous results
on other tunable parameters such as gene copy number and promoter binding
affinity affect gene expression. This gives confidence that with a solid
understanding of the fundamental building blocks, we can accurately model
complex, multi-step systems.