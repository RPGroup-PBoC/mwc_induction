%%%%%%%%%%%%%%%%%%%%%%%%%%%%%%%%%%%%%%%%%%%%%%%%%%%%%%%%%%%%%%%%%%%%%%%%%%%
\section*{\textcolor{Green}{Points do discuss with Rob at next Meeting}}
\begin{itemize}
	\item \robComment{Please remind me about the number of binding sites. Are there
		two binding sites for the dimer or the tetramer?}
	
	\item \robComment{Do we believe that removing the tetramerization region will
		not alter $K_A$, $K_I$, or $e^{\beta \epsilon}$? I understand that within the
		MWC framework it does what you are saying, but the worry has to be that there
		are changes to the protein that are not within the framework of the MWC model.
		We can discuss this further.} \talComment{I feel like such an approximation is
		in line with our thinking on the mutants. It may ultimately be wrong, but let's
		see how far we can go with it.}
\end{itemize}

\section*{\textcolor{blue}{To Do}}

\begin{enumerate}
	\item Stephanie's latest idea: Recast the Franz/Brewster data in terms of the
	summation scheme rather than the neat formula (which does not account for small
	copy numbers of operator or large copy numbers of repressor). See paper
	``Statistical mechanical model of coupled transcription from multiple promoters
	due to transcription factor titration'' \begin{itemize} \item It would be great
		to be able to also use the $x$-measurement error in the fit to Rob's data!
	\end{itemize}
	
	\item Discussion of $\varepsilon$ parameter and the implications of its
	sloppiness. The other parameters are fixed once you nail down $\varepsilon$.
	Introduce Rob Brewster's data  which shows (using Stephanie's version 1 or
	version 2) that it is probably at least $\varepsilon > 0$, but this can not be
	determined exclusively. However, once you choose a parameter value, the other
	parameters will adjust to accommodate this. \begin{enumerate} \item Question:
		Will the mutants data be able to nail this down further? \end{enumerate}
	
	\item \talComment{Thoughts on presenting our O1, O2, O3, Oid data. We should
		distinguish the O2 RBS 1027 curve in some way (perhaps make it dashed) so that
		we can clearly state that: ``only a \textit{single} dataset (O2 RBS 1027,
		dashed) was used to fit the thermodynamic parameters, after which all of the
		remaining titration curves for all other O2 strains as well as all the O1, O3,
		Oid strains became completely constrained predictions without any further
		fitting.``}
	
	\item \talComment{I am personally advocating shoving the model derivation to an
		Appendix. Introducing terms such as $p$, $r_A$, $r_I$ which the reader will
		never see again is just wasting brain space in their attempt to understand the
		paper. We should just quote the definition of fold-change and then skip to the
		result, with the details in an Appendix.}
	
	\item Get proper error on fit parameters. MCMC to get credible region (it would
	be interesting if this is significantly different from the confidence
	interval)?
\end{enumerate}