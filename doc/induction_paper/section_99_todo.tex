%%%%%%%%%%%%%%%%%%%%%%%%%%%%%%%%%%%%%%%%%%%%%%%%%%%%%%%%%%%%%%%%%%%%%%%%%%%
%\section*{\textcolor{Green}{Points do discuss with Rob at next Meeting}}
%\begin{itemize}
%%	\item \robComment{Please remind me about the number of binding sites. Are there
%%		two binding sites for the dimer or the tetramer?}
%%	
%%	\item \robComment{Do we believe that removing the tetramerization region will
%%		not alter $K_A$, $K_I$, or $e^{\beta \epsilon}$? I understand that within the
%%		MWC framework it does what you are saying, but the worry has to be that there
%%		are changes to the protein that are not within the framework of the MWC model.
%%		How much time will this take and is it worth it? We can discuss this further.}
%%	\talComment{I feel like such an approximation is in line with our thinking on
%%		the mutants. It may ultimately be wrong, but let's see how far we can go with
%%		it.}
%	
%	
%\end{itemize}

\pagebreak
\section*{\textcolor{blue}{To Do}}

\begin{enumerate}
	\item \robComment{A very important question coming at all of us from Terry Hwa. Is all of this much less impressive given that each growth condition will have its own story?  We should probably see if we can understand something about growth condition dependence.  However, if we do that, then we have to go back to having ways to count the repressors.  Let's all discuss this though.}
	\manuelComment{I have an interesting experimental proposal simple enough to test that. Remind me with the phrase: ``Hwa's growth laws''}
	
	\item \robComment{Still not happy with sloppiness derivation. Seems you are trying to do two limits (small $\tilde{c}$ and large $\tilde{c}$) simultaneously. \talComment{I am. I am trying to make an approximation that is valid for all $\tilde{c}$ by throwing away the terms that for all $\tilde{c}$ are negligible. And in terms of visually seeing what is a ``good approximation'', that is a personal judgment, since when a function runs from 0 to 1 I would say that  differences smaller than 0.035 may be deemed negligible.}}
	
	\item Stephanie's latest idea: Recast the Franz/Brewster data in terms of the
	summation scheme rather than the neat formula (which does not account for small
	copy numbers of operator or large copy numbers of repressor). See paper
	``Statistical mechanical model of coupled transcription from multiple promoters
	due to transcription factor titration'' \begin{itemize} \item It would be great
		to be able to also use the $x$-measurement error in the fit to Rob's data!
	\end{itemize}
	
	\item Discussion of $\varepsilon$ parameter and the implications of its
	sloppiness. The other parameters are fixed once you nail down $\varepsilon$.
	Introduce Rob Brewster's data  which shows (using Stephanie's version 1 or
	version 2) that it is probably at least $\varepsilon > 0$, but this can not be
	determined exclusively. However, once you choose a parameter value, the other
	parameters will adjust to accommodate this. \begin{enumerate} \item Question:
		Will the mutants data be able to nail this down further? \end{enumerate}
	
	\item \talComment{Thoughts on presenting our O1, O2, O3, Oid data. We should
		distinguish the O2 RBS 1027 curve in some way (perhaps make it dashed) so that
		we can clearly state that: ``only a \textit{single} dataset (O2 RBS 1027,
		dashed) was used to fit the thermodynamic parameters, after which all of the
		remaining titration curves for all other O2 strains as well as all the O1, O3,
		Oid strains became completely constrained predictions without any further
		fitting.``}
	
	\item \talComment{I am personally advocating shoving the model derivation to an
		Appendix. Introducing terms such as $p$, $r_A$, $r_I$ which the reader will
		never see again is just wasting brain space in their attempt to understand the
		paper. We should just quote the definition of fold-change and then skip to the
		result, with the details in an Appendix.}
	
	\item Get proper error on fit parameters. MCMC to get credible region (it would
	be interesting if this is significantly different from the confidence
	interval)?
\end{enumerate}