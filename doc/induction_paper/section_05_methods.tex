\pagebreak
%%%%%%%%%%%%%%%%%%%%%%%%%%%%%%%%%%%%%%%%%%%%%%%%%%%%%%%%%%%%%%%%%%%%%%%%%%%
\section*{Methods \griffinComment{Griffin's Section!}}

\talComment{This section is wicked!}

\subsection*{Bacterial Strains and DNA Constructs}
All strains used in these experiments were adapted from those created and
described in \citep{Garcia2011} \talComment{Let's discuss cite vs citep}. Briefly, the operator mutants and YFP reporter
gene were cloned into a pZS25 background which contains a \textit{lacUV5}
promoter which drives expression as is shown in \ref{fig:S1}. The ribosomal
binding site mutants were cloned into a pZS3\*5 plasmid backbone \griffinComment{find plasmid nomenclature reference} \nathanComment{\cite{Lutz1997} ?}. Expression of
the \textit{lacI} transcript was driven by a P$_\mathrm{LtetO-1}$ promoter
\cite{Lutz1997}. The ribosomal binding site mutants were made as described in
\citep{Salis2009} using site-directed mutagenesis (Quickchange II; Stratagene)
and is described in detail in \citep{Garcia2011}.

A subset of strains in these experiments were measured using fluorescence
microscopy for validation of the flow cytometry data and results. To aid in the
high-fidelity segmentation of individual cells, the above strains were
modified to constitutively express an mCherry fluorophore. This reporter was
cloned into a pZS4*1 backbone in which mCherry is driven by the \textit{lacUV5} promoter.
All microscopy and flow-cytometry experiments were performed using these
strains. For detailed information on plasmid construction, please see the
supplementary information and supplemental table \ref{table:S1}.


\subsection*{Growth conditions for flow cytometry measurements.}
All measurements unless otherwise noted were peformed on \textit{E. coli} cells
grown to mid-exponential phase in standard M9 minimal media supplemented with
0.5\% (w/v) glucose. Briefly, overnight 500mL cultures of Lysogeny Broth (LB
Miller Powder, BD Medical) \textit{E. coli} strains (as is described previously
in section [XX]), were inoculated from a 50\% glycerol frozen stock
(-80$^\circ$) and were grown overnight in a 2mL 96-deep-well plate sealed with
a breatheable nylon cover (Lab Pak - Nitex Nylon, Sefar America Inc.) with
rapid agitation for proper aeration. After approximately 12 - 15 hours, the
cultures had reached saturation and were diluted 1000-fold into 500mL of M9
minimal media containing 0.5\% w/v glucose (anhydrous D-Glucose, Macron
Chemicals) and the appropriate concentration of IPTG (Isopropyl $\beta$-D-1
thiogalactopyranoside Dioxane Free, Resarch products International), were
sealed with a breatheable cover, and were allowed to grow for approximately
eight hours unless otherwise noted. Cells were then diluted ten-fold into a
round-bottom 96-well plates (XXX) containing 90µL of M9 minimal media
supplemented with 0.5\% w/v glucose along with the corresponding IPTG
concentrations. A stock of 100-fold concentrated IPTG in double-distiled water
was prepared and partitioned into 100\textmu L aliquots. The same parent stock
was used for all experiments described in this work.

\subsection*{Single-Cell Microscopy}
Previous studies which make quantitative measurements of fold-change \talComment{we should decide whether to uniformly hyphenate this or not} in gene
expression have typically been performed using single-cell microscopy
\cite{Brewster2014, Jones2014} or bulk-measurment \talComment{spell check} methods such as the LacZ
Miller assay or plate-reader fluorescence measurements \cite{Garcia2011,
Razo-Mejia2014}. However, to our knowledge, no such measurement has been made
with high-throughput single-cell  measurement techniques such as flow-cytometry.
To confirm our flow-cytometry results, we performed a subset of the previously
described experiments using  single-cell microscopy in conjunction with flow-
cytometry.

The data analysis pipeline is described in detail in the supplementary text. In
short, the bacterial strains were grown in the same manner as described
previously. However, immediately before measurement using flow-cytometry, a
small aliquot of cell suspension (less than 10\textmu L) was removed and was
added to a 2\% (w/v) agarose immobilization substrate composed of M9 minimal
media supplemented with 0.5\% (w/v) glucose. The immobilized cells were
attached to a 1mm glass-bottom microscopy dish (Wilco Dish, Ted Pella USA) and
imaged on a Nikon TI-Eclipse widefield epifluorescence microscope.

\griffinComment{Make note of how excitation wavelengths were generated and how
a uniform field was generated. This is probably better for the SI.}

\subsection*{Flow cytometry}
Unless explicitly mentioned, all fold-change measurements were collected on a
Miltenyi Biotec MACSquant Analyzer 10 Flow Cytometer graciously provided by the
Pamela Bj\"{o}rkman lab at Caltech. Detailed information regarding the voltage
and gain  settings of the photo-multiplier detectors can be found in
supplementary table \ref{table:S2}. Prior to each days experiments, the
analyzer was calibrated using MACSQuant Calibration Beads (CAT NO. 130-093-607)
such that day-to-day experiments could be comparable. All YFP fluorescence
measurements were collected via 488nm laser excitation  coupled with a 525/50
nm emission filter. Unless specified, all measurements were  taken over the
course of two to three hours using automated sampling from a 96-well plated
kept at approximately 4$^\circ$ - 10$^\circ$ C \griffinComment{Include the
ice-block-thing name}. Cells were diluted to a final concentration of
approximately 4$\times 10^{4}$ cells per microliter which corresponded to a
flow rate of 5,000 - 10,000 measurements per second \stephanieComment{I think it was more like 2000-6000 on average}. Once completed, the data
was extracted and immediately  processed. \talComment{It may be fun to state explicitly the total number of raw data points per IPTG concentration per run}

\subsection*{Unsupervised gating of flow cytometry data}
The process of restricting the collected data set to those data determined to
be ``real'' is commonly referred to as gating. These gates are typically drawn
manually \cite{Maecker2005} and restrict the data set to those data \talComment{awk} which
display a high degree of linear correlation  between their forward-scatter
(FSC) and side-scatter (SSC). The development of unbiased and unsupervised
methods of drawing these gates is an active area of research
\cite{Agaheepour2013, Lo2008}. In these methods \talComment{it is a bit hard to understand that ``these methods'' does not refer to our project}, however, the rare events are
often the subject of interest. For our purposes, we assume that the
fluorescence level of the population should be normally distributed about some
mean value. With this  assumption in place, we developed a method that allows
us to restrict the data used  to compute the mean fluorescence intensity of the
population to smallest two-dimensional  region of the $\log(\mathrm{FSC})$ vs.
$\log(\mathrm{SSC})$ space in which 40\% of the data is found \talComment{This is confusing}. This  was
performed by fitting a bivariate Gaussian distribution and restricting the
data used for calculation as those that reside within the 40th percentile.
\griffinComment{Let's make sure that the Jupyter notebook for this gating is
crystal clear and well rendered. I feel that this is going to be a serious
sticking point for any fc expert reviewers.} This procedure is  described in
more detail in the supplementary information as well as in a  Jupyter notebook
located on this paper's GitHub repository
\texttt{https://www.github.com/rpgroup-pboc/mwc\_induction/analysis/unsupervised\_gating.ipynb} \talComment{this could be a hyperlink}.

\subsection*{Bayesian parameter estimation}
In this work, we determined the maximum a posteriori (MAP) value for
the inducer binding constants $K_A$ and $K_I$ of the active and inactive
repressors respectively \talComment{what about the $\epsilon$ parameter. We should mention that we fixed the $\epsilon$ value and reference the sloppiness Appendix section}. To compute the MAP, we used a Bayesian method of
inference and determined the probability \talComment{distribution for } \sout{of} each parameter \sout{having a given
value}. Using Bayes's theorem, we can state that the posterior probability
distribution of our parameter of interest is

\begin{align}
P(\epsilon_A, \epsilon_I, R, \Delta\epsilon_r, \vert D, I) \propto P(D \vert
\epsilon_A, \epsilon_I, R, \Delta\epsilon_r, I)\cdot P(\epsilon_A, \epsilon_I,
R, \Delta\epsilon_r \vert I)
\end{align}

in \talComment{punctuation and link sentence} which $\epsilon_A = -\ln K_A$ and
$\epsilon_I = -\ln K_I$ for convenience \talComment{I would actually call this fit stability, since $K_A$ and $K_I$ must be positive}, $R$ is the \sout{mean measurement of the} average number of
repressors per cell, $\Delta\epsilon_r$ is the \sout{average} operator binding energy, $D$ is the experimental data, and $I$ is all prior
information. We can make the assumption that each measurement of fold-change
is independent, the errors on these measurements are Gaussian distributed, and
the error distribution of the concentration of inducer is homoscedastic. Given
these assumptions, it is important to realize that each element of the
experimental measurements $D$ is actually a pair of a dependent variable (the
experimentally measured fold-change) and the independent variables ($R$, the
operator binding energy $\Delta\epsilon_r$, and the IPTG concentration $C$ \talComment{Why capitalize $c$ here?}). Using these assumptions, we can state our likelihood $P(D \vert \epsilon_A, \epsilon_i, R, I)$ as

\begin{align}
P(D \vert \epsilon_A, \epsilon_I, R, \Delta\epsilon_r, I) =
\frac{1}{(2\pi\sigma^2)^{\frac{n}{2}}}\prod\limits_{i=1}^n\exp\left[\frac{(\mathrm{fc_{exp}}^i -
\mathrm{fc}(\epsilon_A, \epsilon_I, R^i, \Delta\epsilon_r^i,
C^i))^2}{2\sigma^2}\right]
\end{align}

where \talComment{link and punctuation, both here and below} $\sigma$ is the variance associated with the Gaussian distributed error
of the MAP, $\mathrm{fc}$ is the the fold-change expression \talComment{why did you use $\mathrm{fc_{exp}}^i$ and not $\mathrm{fc_{exp}^i}$? Note that you used subsuperscripts in $\Delta\epsilon_r^i$} derived previously
\griffinComment{reference the main text equation here once we converge on
notation}, $n$ is the number of experimental measurements, and $i$ is each
individual measurement. We can assume that the parameters $\epsilon_A$,
$\epsilon_I$, $\sigma$ are independent and are uniform on a logarithmic scale.
For this reason, we can assign a Jeffery's prior on these parameters yielding

\begin{align}
P(\epsilon_A, \epsilon_I, \sigma \vert I) \equiv
\frac{1}{\epsilon_A}\cdot\frac{1}{\epsilon_I}\cdot\frac{1}{\sigma}
\end{align}

These priors are maximally uninformative meaning that we imply that we have no
prior knowledge of their actual value. For this work, we can further simplify
our model by assuming that the values for $R$ and $\Delta\epsilon_r$ are fixed
and their values are those reported in \citep{Garcia2011}. Our complete posterior probability distribution can then be written as

\begin{align}
P(\epsilon_A, \epsilon_I, \sigma \vert D, I) \propto
\frac{1}{(2\pi\sigma^2)^\frac{n}{2}}\prod\limits_{i =
1}^n\exp\left[\frac{(\mathrm{fc_{exp}}^i - \mathrm{fc}(\epsilon_A, \epsilon_I,
R^i, \Delta\epsilon_r,^i,
C^i))^2}{2\sigma^2}\right]\frac{1}{\epsilon_A\epsilon_I\sigma}.
\end{align}

However, the values reported for $R$ and $\Delta\epsilon_r$ as
were measured experimentally and by fitting to a predictor strain in  \citep{Garcia2011} respectively. In this work, Garcia and Phillips reported a
mean and standard deviation for the number of repressor molecules per cell. It
is important to note that this does not represent the single-cell variability
of repressor copy number but is rather a measure of uncertainty of this
measurement. We invite the reader to consult the supplementary information the
the propagation of the error into a framework in which we globally fit all
parameters.

With this posterior in hand, we used Markov Chain Monte Carlo to sample our
posterior distribution in log space and compute the MAP value and uncertainty
for the parameter values. \griffinComment{Once we finish collecting data and
perform the final parameter estimation, we should report some of the MCMC
parameters here as well as which sampler was used. I'm in favor of storing the
MCMC flat chains on the git repository so anyone can fork the repository and
immediately and exactly perform our measurements.}

\subsection*{Data curation}
All data was collected, stored, and preserved using the Git version control
software in combination with off-site storage and hosting website GitHub. Code
used to generate all figures and complete all processing step as and analyses
are available on the GitHub repository. Many analysis files are stored as
instructive Jupyter Notebooks. The scientific community is invited to fork our
repositories and open constructive issues on the repository which can be
located at \texttt{https://www.github.com/rpgroup-pboc/mwc\_induction}.

Raw flow cytometry data files (\texttt{.fcs} and \texttt{.csv}) files were
stored on-site under redundant storage. Due to size limitations, all raw data
files are available upon request and are not publicly hosted.

%%%%%%%%%%%%%%%%%%%%%%%%%%%%%%%%%%%%%%%%%%%%%%%%%%%%%%%%%%%%%%%%%%%%%%%%%%%%%%%%
\section*{Acknowledgments}

We thank...for their useful discussions and helpful insights.

This work was supported by...
